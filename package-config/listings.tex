\definecolor{listingsstringcolor}{rgb}{0,0.5,0}
\definecolor{listingskeywordcolor}{rgb}{0.5,0.5,0.0}
\definecolor{listingsbasiccolor}{rgb}{0.0,0.0,0.0}
% \definecolor{listingskeywordcolor}{rgb}{0.0,0.0,0.7}
\definecolor{listingsnumbercolor}{rgb}{0.0,0.0,1.0}
\definecolor{listingscommentcolor}{rgb}{0.4,0.4,0.4}
\definecolor{listingsbackgroundcolor}{rgb}{0.975,0.975,0.975}
\definecolor{listingsrulecolor}{rgb}{0.86,0.86,0.86}
\definecolor{listingsidentifiercolor}{rgb}{0.0,0.0,0.0}
\definecolor{listingsclasscolor}{rgb}{0.5,0.0,0.5}
\definecolor{listingsmembercolor}{rgb}{0.5,0.0,0.0}
\definecolor{listingsdirectivecolor}{rgb}{0.0,0.0,0.5}
% \definecolor{listingsvariablecolor}{rgb}{0.5,0.0,0.5}

\RequirePackage{iftex}

\newcommand{\listingsfont}{\sffamily}
\lstset {
    language=c++,
    numbers=none,
    breaklines=true,
    tabsize=2,
    backgroundcolor=\color{listingsbackgroundcolor},
    breakatwhitespace=true,         % sets if automatic breaks should only happen at whitespace
    breaklines=true,                 % sets automatic line breaking
  %   numbers=left,                    % where to put the line-numbers; possible values are (none, left, right)
    numbersep=5pt,                   % how far the line-numbers are from the code
    frame=lrtb,                    % adds a frame around the code
    framexleftmargin=7pt,
    framexrightmargin=7pt,
    framextopmargin=7pt,
    framexbottommargin=7pt,
    xleftmargin=18pt,
    xrightmargin=18pt,
    rulecolor=\color{listingsrulecolor},
    tabsize=4,                       % sets default tabsize to 2 spaces
    literate={-}{{\textendash}}1 {å}{{\aa}}1 {æ}{{\ae}}1 {ø}{{\oslash}}1,
    showstringspaces=false,
    captionpos=b,
    basicstyle=\color{listingsbasiccolor}\footnotesize\listingsfont,
    keywordstyle=\color{listingskeywordcolor}\footnotesize\listingsfont,
    directivestyle=\color{listingsdirectivecolor}\footnotesize\listingsfont,
    stringstyle=\color{listingsstringcolor}\footnotesize\listingsfont,
    commentstyle=\color{listingscommentcolor}\footnotesize\listingsfont,
    numberstyle=\color{listingsnumbercolor}\footnotesize\listingsfont,
    identifierstyle=\color{listingsidentifiercolor}\footnotesize\listingsfont,
    keywordstyle=[2]{\color{listingsclasscolor}\footnotesize\listingsfont},
    keywordstyle=[3]{\color{listingsmembercolor}\footnotesize\listingsfont},
    keywordstyle=[4]{\color{listingsdirectivecolor}\footnotesize\listingsfont},
}
\lstdefinelanguage[std]{c++}[ISO]{c++}
{
    morekeywords=[2]{
        array,deque,forward_list,list,map,queue,set,stack,unordered_map,unordered_set,vector,
        unordered_multimap,multimap,multiset,
        basic_istringstream,basic_ostringstream,basic_stringstream,basic_stringbuf,
        istringstream,ostringstream,stringstream,stringbuf,wstringstream,wostringstream,wstringstream,wstringbuf,
        fstream,iomanip,ios,iosfwd,istream,ostream,sstream,streambuf,
        basic_ios,fpos,ios_base,io_errc,streamoff,streampos,streamsize,wstreampos,
        promise,packaged_task,future,shared_future,future_error,future_errc,future_status,launch,
        basic_ifstream,basic_ofstream,basic_fstream,basic_filebuf,
        ifstream,ofstream,fstream,filebuf,wifstream,wofstream,wfstream,wfilebuf,
        atomic,condition_variable,future,mutex,thread,
        algorithm,
        bitset,
        chrono,
        codecvt,
        complex,
        exception,
        functional,
        initializer_list,
        iterator,
        limits,
        locale,
        memory,
        new,
        numeric,
        random,
        ratio,
        regex,smatch,
        stdexcept,
        string,
        system_error,
        tuple,
        typeindex,
        typeinfo,
        type_traits,
        utility,
        valarray,
        System,
        Atom,
        HartreeFockBasisParser,
        Sampler,
        Hamiltonian,
        WaveFunction,
        Orbital,
        SlaterWithJastrow,
        Electron,
        Metropolis,
        Core,
        vec,
        SlaterTypeOrbital,
        GaussianOrbital,
        HydrogenOrbital,
        mat,
        RestrictedHartreeFock,
        UnrestrictedHartreeFock,
        Hydrogen,
        Helium,
        Lithium,
        Beryllium,
        Boron,
        Carbon,
        Nitrogen,
        Oxygen,
        Flourine,
        Neon,
        GaussianPrimitive,
        HermiteGaussian,
        OverlapIntegrator,
        ContractedIntegrator,
        ContractedGaussian,
        cube,
        arma,
        KineticIntegrator,
        BoysFunction,
        zeros,
        boost,
        tgamma_lower,
        HermiteGaussianIntegral,
        field,
        cube,
        ElectronNucleusIntegrator,
        ElectronElectronIntegrator,
        eig_sym,
        Orbital,
        HydrogenOrbital,
        GaussianOrbital,
        SlaterTypeOrbital,
        PrimitiveGaussian,
        ContractedGaussian,
        Metropolis,
        Sampler,
        System,
        Hamiltonian,
        std,
        normal_distribution,
        uniform_int_distribution,
        uniform_real_distribution,
    },
    morekeywords=[3]{
        m_system,
        m_slaterUp,
        m_slaterDown,
        m_slaterUpOld,
        m_slaterDownOld,
        m_slaterGradientUp,
        m_slaterGradientDown,
        m_slaterLaplacianUp,
        m_slaterLaplacianDown,
        m_slaterGradientUpOld,
        m_slaterGradientDownOld,
        m_spinChanged,
        m_orbital,
        m_Rsd,
        m_Rc,
        m_numberOfElectrons,
        m_numberOfSpinUpElectrons,
        m_numberOfSpinDownElectrons,
        m_Ex,
        m_Ey,
        m_Ez,
        m_sqrtPiOverP,
        m_hermiteGaussian,
        m_overlapIntegrator,
        m_coefficients,
        m_exponent1,
        m_exponent2,
        m_nucleusPosition1,
        m_nucleusPosition2,
        m_primitive1,
        m_primitive2,
        m_T,
        m_overlapIntegrals,
        m_recurrenceValues,
        m_boysFunction,
        m_t,
        m_u,
        m_v,
        m_PC,
        m_tuv,
        m_hermiteGaussian12,
        m_hermiteGaussian34,
        m_overlapMatrix,
        m_fockMatrixTilde,
        m_fockMatrix,
        m_numberOfBasisFunctions,
        m_numberOfElectrons,
        m_epsilon,
        m_coefficientMatrix,        
        m_coefficientMatrixTilde,
        m_oneBodyMatrixElements,
        m_nucleusNucleusInteractionEnergy,
        m_densityMatrix,
        m_oneBodyMatrixElements,
        m_hartreeFockEnergy,
        m_transformationMatrix,
        m_smoothingFactor,
        m_smoothing,
        m_interElectronDistances,
        m_beta,
        m_spinMatrix,
        m_correlationMatrixOld,
        m_correlationMatrix,
        m_changedElectron,
        m_electronPositions,
        m_jastrowLaplacianTerms,
        m_jastrowGradient,
        m_jastrowLaplacian,
        m_energyCrossTerm,
        m_quantumForce,
        m_alpha,
        m_2sNormalization,
        m_currentValue,
        m_primitives,
        m_basis,
        m_spinDownCoefficients,
        m_spinUpCoefficients,
        m_alpha,
        m_i,
        m_j,
        m_k,
        m_constant,
        m_basisSize,
        m_system,
        m_waveFunction,
        m_numberOfDimensions,
        m_numberOfElectrons,
        m_samplers,
        m_dt,
        m_dtSqrt,
        m_stepLength,
        m_stepLengthHalf,
        m_importanceSampling,
        m_stepLengthSetManually,
        m_numberOfMetropolisSteps,
        m_silent,
        m_randomGenerator,
    },
    morecomment=[l][keywordstyle4]{\#include},
}


\lstdefinelanguage[std]{Python}
{
    morekeywords=[2]{
        array,deque,forward_list,list,map,queue,set,stack,unordered_map,unordered_set,vector,
        unordered_multimap,multimap,multiset,
        basic_istringstream,basic_ostringstream,basic_stringstream,basic_stringbuf,
        istringstream,ostringstream,stringstream,stringbuf,wstringstream,wostringstream,wstringstream,wstringbuf,
        fstream,iomanip,ios,iosfwd,istream,ostream,sstream,streambuf,
        basic_ios,fpos,ios_base,io_errc,streamoff,streampos,streamsize,wstreampos,
        promise,packaged_task,future,shared_future,future_error,future_errc,future_status,launch,
        basic_ifstream,basic_ofstream,basic_fstream,basic_filebuf,
        ifstream,ofstream,fstream,filebuf,wifstream,wofstream,wfstream,wfilebuf,
        atomic,condition_variable,future,mutex,thread,
        algorithm,
        bitset,
        chrono,
        codecvt,
        complex,
        exception,
        functional,
        initializer_list,
        iterator,
        limits,
        locale,
        memory,
        new,
        numeric,
        random,
        ratio,
        regex,smatch,
        stdexcept,
        string,
        system_error,
        tuple,
        typeindex,
        typeinfo,
        type_traits,
        utility,
        valarray,
        System,
        Atom,
        HartreeFockBasisParser,
        Sampler,
        Hamiltonian,
        WaveFunction,
        Orbital,
        SlaterWithJastrow,
        Electron,
        Metropolis,
        Core,
        vec,
        SlaterTypeOrbital,
        GaussianOrbital,
        HydrogenOrbital,
        mat,
        RestrictedHartreeFock,
        UnrestrictedHartreeFock,
        Hydrogen,
        Helium,
        Lithium,
        Beryllium,
        Boron,
        Carbon,
        Nitrogen,
        Oxygen,
        Flourine,
        Neon,
        GaussianPrimitive,
        HermiteGaussian,
        OverlapIntegrator,
        ContractedIntegrator,
        ContractedGaussian,
        cube,
        arma,
        KineticIntegrator,
        BoysFunction,
        zeros,
        boost,
        tgamma_lower,
        HermiteGaussianIntegral,
        field,
        cube,
        ElectronNucleusIntegrator,
        ElectronElectronIntegrator,
        eig_sym,
        Orbital,
        HydrogenOrbital,
        GaussianOrbital,
        SlaterTypeOrbital,
        PrimitiveGaussian,
        ContractedGaussian,
        Metropolis,
        Sampler,
        System,
        Hamiltonian,
        std,
        normal_distribution,
        uniform_int_distribution,
        uniform_real_distribution,
    },
    morekeywords=[3]{
        m_system,
        m_slaterUp,
        m_slaterDown,
        m_slaterUpOld,
        m_slaterDownOld,
        m_slaterGradientUp,
        m_slaterGradientDown,
        m_slaterLaplacianUp,
        m_slaterLaplacianDown,
        m_slaterGradientUpOld,
        m_slaterGradientDownOld,
        m_spinChanged,
        m_orbital,
        m_Rsd,
        m_Rc,
        m_numberOfElectrons,
        m_numberOfSpinUpElectrons,
        m_numberOfSpinDownElectrons,
        m_Ex,
        m_Ey,
        m_Ez,
        m_sqrtPiOverP,
        m_hermiteGaussian,
        m_overlapIntegrator,
        m_coefficients,
        m_exponent1,
        m_exponent2,
        m_nucleusPosition1,
        m_nucleusPosition2,
        m_primitive1,
        m_primitive2,
        m_T,
        m_overlapIntegrals,
        m_recurrenceValues,
        m_boysFunction,
        m_t,
        m_u,
        m_v,
        m_PC,
        m_tuv,
        m_hermiteGaussian12,
        m_hermiteGaussian34,
        m_overlapMatrix,
        m_fockMatrixTilde,
        m_fockMatrix,
        m_numberOfBasisFunctions,
        m_numberOfElectrons,
        m_epsilon,
        m_coefficientMatrix,        
        m_coefficientMatrixTilde,
        m_oneBodyMatrixElements,
        m_nucleusNucleusInteractionEnergy,
        m_densityMatrix,
        m_oneBodyMatrixElements,
        m_hartreeFockEnergy,
        m_transformationMatrix,
        m_smoothingFactor,
        m_smoothing,
        m_interElectronDistances,
        m_beta,
        m_spinMatrix,
        m_correlationMatrixOld,
        m_correlationMatrix,
        m_changedElectron,
        m_electronPositions,
        m_jastrowLaplacianTerms,
        m_jastrowGradient,
        m_jastrowLaplacian,
        m_energyCrossTerm,
        m_quantumForce,
        m_alpha,
        m_2sNormalization,
        m_currentValue,
        m_primitives,
        m_basis,
        m_spinDownCoefficients,
        m_spinUpCoefficients,
        m_alpha,
        m_i,
        m_j,
        m_k,
        m_constant,
        m_basisSize,
        m_system,
        m_waveFunction,
        m_numberOfDimensions,
        m_numberOfElectrons,
        m_samplers,
        m_dt,
        m_dtSqrt,
        m_stepLength,
        m_stepLengthHalf,
        m_importanceSampling,
        m_stepLengthSetManually,
        m_numberOfMetropolisSteps,
        m_silent,
        m_randomGenerator,
    },
    morecomment=[l][keywordstyle4]{\#include},
}


\input{package-config/listings-qmake.tex}
\input{package-config/listings-glsl.tex}
\lstdefinelanguage[std]{c++}[ISO]{c++}
{
    morekeywords=[2]{
        array,deque,forward_list,list,map,queue,set,stack,unordered_map,unordered_set,vector,
        unordered_multimap,multimap,multiset,
        basic_istringstream,basic_ostringstream,basic_stringstream,basic_stringbuf,
        istringstream,ostringstream,stringstream,stringbuf,wstringstream,wostringstream,wstringstream,wstringbuf,
        fstream,iomanip,ios,iosfwd,istream,ostream,sstream,streambuf,
        basic_ios,fpos,ios_base,io_errc,streamoff,streampos,streamsize,wstreampos,
        promise,packaged_task,future,shared_future,future_error,future_errc,future_status,launch,
        basic_ifstream,basic_ofstream,basic_fstream,basic_filebuf,
        ifstream,ofstream,fstream,filebuf,wifstream,wofstream,wfstream,wfilebuf,
        atomic,condition_variable,future,mutex,thread,
        algorithm,
        bitset,
        chrono,
        codecvt,
        complex,
        exception,
        functional,
        initializer_list,
        iterator,
        limits,
        locale,
        memory,
        new,
        numeric,
        random,
        ratio,
        regex,smatch,
        stdexcept,
        string,
        system_error,
        tuple,
        typeindex,
        typeinfo,
        type_traits,
        utility,
        valarray,
        System,
        Atom,
        HartreeFockBasisParser,
        Sampler,
        Hamiltonian,
        WaveFunction,
        Orbital,
        SlaterWithJastrow,
        Electron,
        Metropolis,
        Core,
        vec,
        SlaterTypeOrbital,
        GaussianOrbital,
        HydrogenOrbital,
        mat,
        RestrictedHartreeFock,
        UnrestrictedHartreeFock,
        Hydrogen,
        Helium,
        Lithium,
        Beryllium,
        Boron,
        Carbon,
        Nitrogen,
        Oxygen,
        Flourine,
        Neon,
        GaussianPrimitive,
        HermiteGaussian,
        OverlapIntegrator,
        ContractedIntegrator,
        ContractedGaussian,
        cube,
        arma,
        KineticIntegrator,
        BoysFunction,
        zeros,
        boost,
        tgamma_lower,
        HermiteGaussianIntegral,
        field,
        cube,
        ElectronNucleusIntegrator,
        ElectronElectronIntegrator,
        eig_sym,
        Orbital,
        HydrogenOrbital,
        GaussianOrbital,
        SlaterTypeOrbital,
        PrimitiveGaussian,
        ContractedGaussian,
        Metropolis,
        Sampler,
        System,
        Hamiltonian,
        std,
        normal_distribution,
        uniform_int_distribution,
        uniform_real_distribution,
    },
    morekeywords=[3]{
        m_system,
        m_slaterUp,
        m_slaterDown,
        m_slaterUpOld,
        m_slaterDownOld,
        m_slaterGradientUp,
        m_slaterGradientDown,
        m_slaterLaplacianUp,
        m_slaterLaplacianDown,
        m_slaterGradientUpOld,
        m_slaterGradientDownOld,
        m_spinChanged,
        m_orbital,
        m_Rsd,
        m_Rc,
        m_numberOfElectrons,
        m_numberOfSpinUpElectrons,
        m_numberOfSpinDownElectrons,
        m_Ex,
        m_Ey,
        m_Ez,
        m_sqrtPiOverP,
        m_hermiteGaussian,
        m_overlapIntegrator,
        m_coefficients,
        m_exponent1,
        m_exponent2,
        m_nucleusPosition1,
        m_nucleusPosition2,
        m_primitive1,
        m_primitive2,
        m_T,
        m_overlapIntegrals,
        m_recurrenceValues,
        m_boysFunction,
        m_t,
        m_u,
        m_v,
        m_PC,
        m_tuv,
        m_hermiteGaussian12,
        m_hermiteGaussian34,
        m_overlapMatrix,
        m_fockMatrixTilde,
        m_fockMatrix,
        m_numberOfBasisFunctions,
        m_numberOfElectrons,
        m_epsilon,
        m_coefficientMatrix,        
        m_coefficientMatrixTilde,
        m_oneBodyMatrixElements,
        m_nucleusNucleusInteractionEnergy,
        m_densityMatrix,
        m_oneBodyMatrixElements,
        m_hartreeFockEnergy,
        m_transformationMatrix,
        m_smoothingFactor,
        m_smoothing,
        m_interElectronDistances,
        m_beta,
        m_spinMatrix,
        m_correlationMatrixOld,
        m_correlationMatrix,
        m_changedElectron,
        m_electronPositions,
        m_jastrowLaplacianTerms,
        m_jastrowGradient,
        m_jastrowLaplacian,
        m_energyCrossTerm,
        m_quantumForce,
        m_alpha,
        m_2sNormalization,
        m_currentValue,
        m_primitives,
        m_basis,
        m_spinDownCoefficients,
        m_spinUpCoefficients,
        m_alpha,
        m_i,
        m_j,
        m_k,
        m_constant,
        m_basisSize,
        m_system,
        m_waveFunction,
        m_numberOfDimensions,
        m_numberOfElectrons,
        m_samplers,
        m_dt,
        m_dtSqrt,
        m_stepLength,
        m_stepLengthHalf,
        m_importanceSampling,
        m_stepLengthSetManually,
        m_numberOfMetropolisSteps,
        m_silent,
        m_randomGenerator,
    },
    morecomment=[l][keywordstyle4]{\#include},
}


\input{package-config/listings-arma.tex}
\input{package-config/listings-qt.tex}
\input{package-config/listings-qml.tex}

% \lstset{
%     literate={0}{{\textcolor{listingsnumbercolor}{0}}}{1}%
%              {1}{{\textcolor{listingsnumbercolor}{1}}}{1}%
%              {2}{{\textcolor{listingsnumbercolor}{2}}}{1}%
%              {3}{{\textcolor{listingsnumbercolor}{3}}}{1}%
%              {4}{{\textcolor{listingsnumbercolor}{4}}}{1}%
%              {5}{{\textcolor{listingsnumbercolor}{5}}}{1}%
%              {6}{{\textcolor{listingsnumbercolor}{6}}}{1}%
%              {7}{{\textcolor{listingsnumbercolor}{7}}}{1}%
%              {8}{{\textcolor{listingsnumbercolor}{8}}}{1}%
%              {9}{{\textcolor{listingsnumbercolor}{9}}}{1}%
%              {.0}{{\textcolor{listingsnumbercolor}{.0}}}{2}% Following is to ensure that only periods
%              {.1}{{\textcolor{listingsnumbercolor}{.1}}}{2}% followed by a digit are changed.
%              {.2}{{\textcolor{listingsnumbercolor}{.2}}}{2}%
%              {.3}{{\textcolor{listingsnumbercolor}{.3}}}{2}%
%              {.4}{{\textcolor{listingsnumbercolor}{.4}}}{2}%
%              {.5}{{\textcolor{listingsnumbercolor}{.5}}}{2}%
%              {.6}{{\textcolor{listingsnumbercolor}{.6}}}{2}%
%              {.7}{{\textcolor{listingsnumbercolor}{.7}}}{2}%
%              {.8}{{\textcolor{listingsnumbercolor}{.8}}}{2}%
%              {.9}{{\textcolor{listingsnumbercolor}{.9}}}{2}%
% }

\RequirePackage{calc}

%Define a reference depth. 
%You can choose either relative or absolute.
%--------------------------
\newlength{\DepthReference}
% \settodepth{\DepthReference}{g}%relative to a depth of a letter.
\setlength{\DepthReference}{0pt}%absolute value.

%Define a reference Height. 
%You can choose either relative or absolute.
%--------------------------
\newlength{\HeightReference}
\settoheight{\HeightReference}{T}
%\setlength{\HeightReference}{7pt}

\newlength{\Width}%
\newcommand{\lstinlinebox}[1]{%
  \settowidth{\Width}{\lstinline|#1|}%
  \fcolorbox{listingsrulecolor}{listingsbackgroundcolor}{%
    \raisebox{-\DepthReference}%
    {%
          \parbox[b][\HeightReference+\DepthReference][c]{\Width}{\centering \lstinline|#1|}%
    }%
  }%
}%