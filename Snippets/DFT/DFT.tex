
\documentclass[a4paper]{article}
%Included packages ----------------------------------------------------------%
\usepackage{inputenc}                        % utf-8 encoding, æ, ø , å, etc.
\usepackage{a4wide}                          % Adjust margins to better fit A4 format.
\usepackage{array}                           % Matrices.
\usepackage{amsmath}                         % Math symbols, and enhanced matrices.
\usepackage{amsfonts}                        % Math fonts.
\usepackage{amssymb}                         % Additional symbols.
%\usepackage{wasysym}                        % More additional symbols.
\usepackage{mathrsfs}                        % Most additional symbols.
\usepackage[pdftex]{graphicx}                % Improved inclusion of .pdf-graphics files.
\usepackage{sidecap}                         % Floats with captions to the right/left.
\usepackage{cancel}                          % Visualize cancellations in equations.
\usepackage{enumerate}                       % Change counters (arabic, roman, etc.).
\usepackage{units}                           % Adds better looking fractions (nicefrac).
\usepackage{floatrow}                        % Multi-figure floats.
\usepackage{subfig}                          % Multi-figure floats.
\usepackage{caption}                         % Adds functionality to captions.
\usepackage{bm}                              % Bolded text in math mode.
\usepackage{combinedgraphics}                % Figures; let latex handle the text itself.
\usepackage[framemethod=default]{mdframed}   % Make boxes.
\usepackage{listings}                        % For including source code.
\usepackage[colorlinks]{hyperref}            % Interactive references, colored.
\usepackage{soul}                            % Make vertical bars through text.
\usepackage{nicefrac}                        % Nice fractions with \nicefrac.
\usepackage{mathtools}                       % Underbrackets, overbrackets.
\usepackage{wasysym}                         % \smiley{}-s!
\usepackage{multicol}                        % Multiple text columns.
\usepackage{capt-of}                         % Caption things which are not floats.
\usepackage[url=false]{biblatex}             % Citations (made easy).
\usepackage{dsfont}
\usepackage{booktabs}                        % Tables
\usepackage{tabularx}
\usepackage{array}
\usepackage{multirow}% http://ctan.org/pkg/multirow
\usepackage{hhline}% http://ctan.org/pkg/hhline
\usepackage{siunitx}
\usepackage[version=4]{mhchem}




% Differentials -------------------------------------------------------------- %
\newcommand{\dt}{\,\mathrm{d}t}
\newcommand{\dx}{\,\mathrm{d}x}
\newcommand{\dr}{\,\mathrm{d}r}

% Derivatives ---------------------------------------------------------------- %
\newcommand{\der} [2]{\frac{\mathrm{d} #1}{\mathrm{d} #2}}   % Derivative.
\newcommand{\pder}[2]{\frac{\partial   #1}{\partial   #2}}   % Partial derivative.

% Matrices ------------------------------------------------------------------- %
\newcommand{\mat} [2]{\begin{matrix}[#1]  #2 \end{matrix}}   % Nothing enclosing it.
\newcommand{\pmat}[2]{\begin{pmatrix}[#1] #2 \end{pmatrix}}  % Enclosing parentheses.
\newcommand{\bmat}[2]{\begin{bmatrix}[#1] #2 \end{bmatrix}}  % Enclosing square brackets.
\newcommand{\vmat}[2]{\begin{vmatrix}[#1] #2 \end{vmatrix}}  % Enclosing vertical bars.
\newcommand{\Vmat}[2]{\begin{Vmatrix}[#1] #2 \end{Vmatrix}}  % Enclosing double bars.

% Number sets ---------------------------------------------------------------- %
\newcommand{\R}{\mathbb{R}}
\newcommand{\Q}{\mathbb{Q}}
\newcommand{\N}{\mathbb{N}}
\newcommand{\Z}{\mathbb{Z}}
\newcommand{\C}{\mathbb{C}}

% Manually set alignment of rows / columns in matrices (mat, pmat, etc.) ----- %
\makeatletter
\renewcommand*\env@matrix[1][*\c@MaxMatrixCols c]{%
  \hskip -\arraycolsep
  \let\@ifnextchar\new@ifnextchar
  \array{#1}}
\makeatother

% References ----------------------------------------------------------------- %
\newcommand{\Fig}[1]{Fig.\ \ref{fig:#1}}
\newcommand{\fig}[1]{Fig.\ \ref{fig:#1}}
\newcommand{\eq} [1]{Eq.\ (\ref{eq:#1})}
\newcommand{\Eq} [1]{Eq.\ (\ref{eq:#1})}
\newcommand{\tab}[1]{Table \ref{tab:#1}}
\newcommand{\Tab}[1]{Table \ref{tab:#1}}

% Paragraph formatting ------------------------------------------------------- %
\setlength{\parindent}{5.5mm}
\setlength{\parskip}  {0mm}

% Source code listings ------------------------------------------------------- %
\definecolor{commentGreen}{RGB}{34,139,34}
\definecolor{keywordBlue}{RGB}{0,0,255}
\definecolor{stringPurple}{RGB}{160,32,240}
\lstset{language=matlab}
\lstset{basicstyle=\ttfamily\small}
\lstset{frame=single}
\lstset{stringstyle=\color{stringPurple}}
\lstset{keywordstyle=\color{keywordBlue}}
\lstset{commentstyle=\color{commentGreen}}
\lstset{morecomment=[l][\color{commentGreen}\bfseries]{\%\%}}
\lstset{showspaces=false}
\lstset{showstringspaces=false}
\lstset{showtabs=true}
\lstset{columns=fixed}
\lstset{breaklines}
\lstset{literate={~} {$\sim$}{1}}
\lstset{numbers=left}              
\lstset{stepnumber=1}
\renewcommand{\ttdefault}{pcr}
\lstdefinestyle{prt}{frame=none,basicstyle=\ttfamily\small}

% Convenient shorthand notation ---------------------------------------------- %
\newcommand{\nn}{\nonumber}
\newcommand{\e}[1]{\cdot10^{#1}}
\renewcommand{\i}{\hat{\imath}}
\renewcommand{\j}{\hat{\jmath}}
\renewcommand{\k}{\hat{k}}

% Caption position of tables at the top -------------------------------------- %
\floatsetup[table]{capposition=top}

% Black frame with white background ------------------------------------------ %
\newmdenv[linecolor=black,backgroundcolor=white]{exframe}

% Including vector drawings from inkscape ------------------------------------ %
\newenvironment{combFig}[5]{
  \begin{figure}[#1] 
    \centering 
    \includecombinedgraphics[vecscale=#2, keepaspectratio]{#3} 
    \caption{#4 \label{#5}}
  \end{figure}
  }

  {
}

% Including pdf graphics ----------------------------------------------------- %
\newenvironment{pdfFig}[5]{
  \begin{figure}[#1] 
    \centering 
    \includegraphics[width= #2]{#3} 
    \caption{#4 \label{#5}}
  \end{figure}
  }

  {
}

% Exercise and subexercise counters ------------------------------------------ %
\newcounter{excounter}
\renewcommand\theexcounter{\arabic{excounter}}
\newcommand\exlabel{\theexcounter}
\setcounter{excounter}{1}

\newcounter{subexcounter}
\renewcommand\thesubexcounter{\arabic{subexcounter}}
\newcommand\subexlabel{\thesubexcounter}
\setcounter{subexcounter}{1}

% Environments for exercises ------------------------------------------------- %
\newenvironment{exercise}[1]{
  \subsection*{Exercise \theexcounter: #1}
  \setcounter{subexcounter}{1}                      % Reset the subexercise counter to a.
  \addcontentsline{toc}{section}{\theexcounter: #1} % Add the exercise to TOC
  }
      % Exercise text.
  {
  \stepcounter{excounter}                           % Add one to the exercise counter.
  \newpage
}

% Environment for subexercises ----------------------------------------------- %
\newenvironment{subexercise}{
  \begin{exframe}
    \begin{itemize}  \setlength{\itemindent}{1cm}
      \item[{\bf Exercise \thesubexcounter}] 
	}
	  % Subexercise text.
	{
    \end{itemize}
  \end{exframe}
  \stepcounter{subexcounter}                        % Add one to the exercise counter.
}

% Environment for proofs ----------------------------------------------------- %
\newenvironment{proof}[2]{
  \begin{exframe}
    \begin{itemize}  \setlength{\itemindent}{0.6cm}
      \item[{\bf #1} {\bf #2}] 
	}
	  % Subexercise text.
	{
    \end{itemize}
  \end{exframe}
}

% Environment for answers ---------------------------------------------------- %
\newenvironment{answer}{}{}

% Set bibliography file and path for images.
\bibliography{references/fys4180ref.bib}
\graphicspath{{./images/}}
\newcommand{\includepdfgraphics}[2]{\includecombinedgraphics[#1]{./images/#2}}



\graphicspath{{/Users/morten/Documents/Master/Master/Figures/}}



% Title
\title{}
\date{}
\author{}
% ---------------------------------------------------------------------------- %
% ---------------------------------------------------------------------------- %


\begin{document}


\renewcommand{\R}{{\bf R}}
\renewcommand{\r}{{\bf r}}
\newcommand{\p}{{\bf p}}
\newcommand{\q}{{\bf q}}
\renewcommand{\H}{\mathcal{H}}
\newcommand{\psit}{\left|\psi(t)\right\rangle}


Yang and Parr: Density functional theory of atoms and molecules

Atomic Reference Data for Electronic Structure Calculations \url{http://math.nist.gov/DFTdata/atomdata/node4.html}

Vosko,Wilk,Nusair: Accurate spin-dependent electron liquid correlation energies for local spin density calculations: A critical analysis \url{http://www.nrcresearchpress.com/doi/pdf/10.1139/p80-159}

\section{Density functional theory}
%\subsection{Intro and motivation}
Because the wave function is such an unbelievably complicated function, depending on $4N$ degrees of freedom (of which $3N$ are spatial coordinates), it is natural to ask the question: Is it possible to represent the state of an electronic system in a more succinct way? A natural candidate for such an entity is the electronic number density, $\rho(\r)$, which we will mostly refer to as simply \emph{the density}. It would be remarkable if we could deal with enormously complex quantum mechanical systems by means of a function depending only on three spatial coordinates and spin(!) ((Kryachko,Ludena p163)). 

It turns out that exactly this is possible. There is a one-to-one correspondance between the ground state density, $\rho_0(\r)$, and the external potential (up to an additive constant) and thus also the Hamiltonian. ((Toulouse, p5)) Since the Hamiltonian uniquely determines all properties of a quantum mechanical system, we can in principle determine all the information in the many-body wave function (of the ground state and \emph{all} excited states) from the ground state density alone ((Martin, pp119)). The fact that the density can be determined from the wave function is almost trivially true, but that the converse is true is the content of the Hohenberg-Kohn theorems. 

However, the theorems of Hohenberg and Kohn guarantee only the existence and uniqueness of an \emph{energy functional}, $E[\rho]$, which can be used to determine the energy from the density. Without knowing the form of $E[\rho]$ and a computational scheme for calculating it, we are still no closer to being able to use the electron density as the primary quantity in electronic structure calculations. This is where the Kohn-Sham ansatz comes in, making it possible for us to calculate structure properties of electronic systems by essentially solving a different system\----a non-interacting system with the electrons moving in an effective potential which by construction yields the same ground state density as the original system. 

We will begin our discussion of {\bf Density functional theory} (DFT [sometimes prepended KS, making it Kohn-Sham density functional theory \{KS-DFT\}]) by considering the theoretical framework and the Hohenberg-Kohn theorems. Then we will consider the Kohn-Sham ansatz and how DFT calculations are performed in practice.


\subsection{The electronic many-body problem in terms of the electron density}
Recall from ((Born-oppenheimer section)) that the \emph{electronic} Hamiltonian under the Born-Oppenheimer approximation takes the form
\begin{align}
\hat H &= -\frac{1}{2}\sum_{i=1}^N\nabla^2_i + \sum_{i=1}^N\sum_{j=i+1}^N \frac{1}{|\r_i-\r_j|} - \sum_{i=1}^N\sum_{A=1}^M \frac{Z_A}{|\r_i-\r_A|}.
\end{align}
We will in the following relabel the last term, and define
\begin{align}
\sum_{i=1}^{N}\sum_{A=1}^M \frac{-Z_A}{|\r_i-\r_A|} &\equiv \hat V_\text{ext}.
\end{align}


























\subsection{Local density approximation}
\subsubsection{VWN-LDA}
((YangAndParr, pp275)) ((atomic reference, \url{DFTdata/atomdata/node4.html})) Parametrize the exchange correlation terms: After Vosko, Wilk, and Nusair ((VWN)) 

\textbf{Exchange term}:
\begin{align}
\varepsilon_\text{x} (r_s,\xi) = \varepsilon_\text{x}^P(r_s) + \left[\varepsilon_\text{x}^P(r_s) - \varepsilon_\text{x}^F(r_s) \right] f(\xi),
\end{align}
with 
\begin{align}
r_s &\equiv \left( \frac{3}{4\pi \rho(\r))} \right)^{\nicefrac{1}{3}}, \ \ \ \leftarrow \text{ electron gas parameter} \\
%
\xi &\equiv \frac{\rho_\uparrow(\r) - \rho_\downarrow(\r)}{\rho_\uparrow(\r) + \rho_\downarrow(\r)} \ \ \ \leftarrow \text{ spin polarization}.
\end{align}

\begin{align}
\varepsilon_\text{x}^P(r_s) &= - 3 \left( \frac{9}{32\pi^2} \right)^{\nicefrac{1}{3}} \frac{1}{r_s}, \\
%
\varepsilon_\text{x}^F(r_s) &= - 3(2^{\nicefrac{1}{3}}) \left( \frac{9}{32\pi^2} \right)^{\nicefrac{1}{3}} \frac{1}{r_s},
\end{align}
and
\begin{align}
f(\xi) = \frac{(1+\xi)^{\nicefrac{4}{3}} + (1-\xi)^{\nicefrac{4}{3}} -2}{2(2^{\nicefrac{1}{3}} - 1)}.
\end{align}

In the limit of no spin-polarization (restricted), $\varepsilon_\text{x}$ takes the value
\begin{align}
\varepsilon_\text{x}^{\xi=0}(r_s)= -\frac{3}{2}\left(\frac{3 \rho(\r)}{\pi}\right)^{\nicefrac{1}{3}} \ \ \ \ \ \ \ \text{possibly missing a factor } \frac{1}{2} \text{?}
\end{align}

\textbf{Correlation term}:
\begin{align}
\varepsilon_\text{c}(r_s) &= \frac{A}{2}\left\{ \ln\left(\frac{x^2}{X(x)}\right) + \frac{2b}{Q}\tan^{-1}\left(\frac{Q}{2x+b}\right) \right.\nn\\
& \ \ \ \ \ \ \ \left.- \frac{bx_0}{X(x_0)} \left[\ln\left(\frac{(x-x_0)^2}{X(x)} \right) + \frac{2(b+2x_0)}{Q}\tan^{-1}\left(\frac{Q}{2x+b} \right) \right]  \right\},
\end{align}
with 
\begin{align}
x &\equiv \sqrt{r_s}, \\
%
X(x) &\equiv x^2+bx+c, \\
%
Q &\equiv \sqrt{4c-b^2}.
\end{align}

For $\xi=0$ we have $A=0.0621814$, $x_0=-0.409286$, $b=13.0720$, and $c=42.7198$.



\end{document}

% \begin{figure}[p!]
% \centering
% \includegraphics[width=12cm]{<fig>.pdf}
% \caption{\label{fig:1}}
% \end{figure}
 
% \lstinputlisting[firstline=1,lastline=2, float=p!, caption={}, label=lst:1]{<code>.m}

