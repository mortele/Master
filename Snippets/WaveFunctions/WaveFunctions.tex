\documentclass[../../master.tex]{subfiles}

\begin{document}


\renewcommand{\R}{{\bf R}}
\renewcommand{\r}{{\bf r}}
\newcommand{\p}{{\bf p}}
\newcommand{\q}{{\bf q}}
\renewcommand{\H}{\mathcal{H}}
\newcommand{\psit}{\left|\psi(t)\right\rangle}


%Salasnich: Quantum Physics of Light and Matter
%Wheeler,Zurek: Quantum Theory and Measurement
%Born: "Zur Quantenmechanik der Stossvorgänge," Zeitschrift für Physik, 37, 863-67 (1926). Translation to english by Wheeler, Zurek
%Weinberg: lectures on quantum mechanics
%Helgaker++: Molecular Electronic Sctructure Theory
%Gross,Runge,Heinonen: Many-particle Theory
%Kato article: \url{http://refhub.elsevier.com/S0009-2614(13)01451-6/h0095}
%Katriel,Davidson: \url{http://www.pnas.org/content/77/8/4403}
%Hammond: Monte Carlo Methods in Ab Initio Quantum Chemistry
%Weissbluth: Atoms and Molecules

\chapter{Wave functions \label{wavefunctions}}
In ordinary quantum mechanics, the wave function is the primary quantity of interest. It constitutes the \emph{solution} of the Schrödinger equation and encodes within it all information about the state of the isolated quantum system in question. Mathematically speaking, the wave function is the complex valued spatial projection of the abstract state vector which is a unitary vector in some separable Hilbert space \cite{kvaal,salasnich}\comment{p8}\comment{p8}. Formally, it is the solution to the Schrödinger equation, $\hat H \psi_k=E_k\psi_k$, and it has a probabilistic interpretation originally after German physicist M. Born \cite{Born1926} (translated by \cite{wheeler}), which states that the magnitude squared is a probability density, i.e. \cite{weinberg}\comment{p24}
\begin{align}
\mathrm{d}P(\r) = \left|\psi(\r)\right|^2 \mathrm{d}^3\r. 
\end{align} 
The probability $\mathrm{d}P(\r)$ denotes here the probability of finding the particle described by the wave function in an infinitesimal volume $\mathrm{d}^3\r$ around the position $\r$. 

Even though we are unable (in the overwhelming majority of cases) to find closed form solutions to the Schrödinger equation, we may nevertheless write down a set of conditions we know the exact solution must adhere to. In the following section, we will go through properties of the exact wave function which are most relevant for atomic and molecular systems. Thereafter, we will consider the most common bases used to form approximate wave functions for many-body quantum systems.

\section{Properties of the exact wave function}
In the words of Helgaker et. al \cite{helgaker}\comment{p108} (paraphrasing), even though we are forced to make approximations in the solution of the Schrödinger equation, such approximations should not be made in a "haphazard manner." Instead we should try to incorporate as many properties of the exact wave function as feasible in our approximate guess. Some of these properties are incorporated almost without thought, such as anti-symmetry and square integrability. However, there are more subtle ones which may be easily missed without a thorough analysis of the known properties of the exact wave function. We present here an incomplete list of properties and conditions for the exact solution of the Schrödinger equation.



\subsubsection*{Eigenfunction of the number operator, $\hat N$}
The exact wave function is a function of the spatial and spin degrees of freedom of the particles it describes. For an atomic or molecular system, under the Born-Oppenheimer approximation, the wave function depends only parametrically on the positions of the nuclei, $\Psi=\Psi(\r_1,\r_2,\dots,\r_N,\sigma_1,\sigma_2,\dots,\sigma_N;\r_A,\r_B,\dots,\r_C)$. The approximation we choose should thus be an eigenfunction of the number operator, $\hat N|\psi\rangle=N|\psi\rangle$. The number operator is defined under second quantization as $\hat N = \sum_q c_q^\dagger c_q$, where the sum is taken to run over all possible single particle states \cite{kvaal}\comment{p24}. 

\subsubsection*{Totally anti-symmetric under exchange of particles}
A fermionic wave function must be totally anti-symmetric w.r.t. exhange of two particles \cite{griffiths}. In accordance with this, we must choose the approximating wave function to be an eigenfunction of the permutation operator, $\hat P_{ij}$, which interchanges particles $i$ and $j$. We must also demand that the eigenvalue is $-1$, i.e. $\hat P_{ij} |\psi\rangle = -1|\psi\rangle$. 

Both of the aforementioned conditions are satisfied if we take the approximation to be an $N \times N$ (or a linear combination of) Slater determinant(s) filled with single electron orbitals.

\subsubsection*{Square integrability and normalization}
For a bound state, the exact wave function is square integrable and normalized to unity, $\langle \Psi|\Psi\rangle = 1$ \cite{helgaker}\comment{p108}. This means that the exact wave function is finite almost everywhere\footnote{Mathematically, it is finite except possibly on a set of measure zero such that the integral over space is not affected by the divergent value.} w.r.t. the $L^2$ norm. A sufficient and natural way to ensure this holds for the approximating wave function is to build it from finite single-electron orbitals, i.e. populate the Slater determinant(s) with spin-orbitals which are themselves parts of $L^2$. 

\subsubsection*{Size-extensivity}
The exact wave function is size-extensive in that a system of non-interacting subsystems have the same total energy as the sum of the energies of the subsystems \cite{helgaker}\comment{p109}. It is thus reasonable to demand of the the approximate wave function that (in some chosen calculational scheme) the energy found by calculating the total energy of a system of non-interacting subsystems to exactly coincide with the energies of the subsystems themselves. In practice, we may check such a condition by separating subsystems by a large distance and comparing the calculated energy with the energy resulting from calculating the energies of the subsystems individually. 

\subsubsection*{Eigenfunction of the total spin and projected spin operators, $\hat S^2$ and $\hat S_z$}
In non-relativistic theory, the exact wave function is an eigenfunction of the total spin operator $\hat S^2$ and the spin projection operator $\hat S_z$ \cite{gross}\comment{p67}. It would be natural to demand that the approximating wave function also be an eigenfunction of these two operators. Taking the approximation to be a single Slater determinant, populated with spin-orbitals of definite spin projection automatically means the total wave function is an eigenfunction of $\hat S_z$. However, single determinants are not \emph{necessarily} eigenfunctions of $\hat S^2$ \cite{szabo}\comment{p100}. Linear combinations of determinants may be formed which by construction are eigenfunctions of $\hat S^2$ \cite{helgaker}\comment{p51}. Such wave functions are called spin-adapted. 

\subsubsection*{Asymptotic behaviour of the electronic density}
Katriel and Davidson \cite{katriel} showed that the electron density decays exponentially as
\begin{align}
\rho(\r)\approx \exp\left( -2\sqrt{2I}r \right),
\end{align}
in the limit of large $r$. Here $I$ denotes the first ionization potential of the molecule, i.e. the energy needed in order to remove the least tightly bound electron. Since the ionization potential is not known before the solution to the Schrödinger equation is found, an a priori treatment of the long-range exponential decay of the density is impossible \cite{helgaker}\comment{p110}. 

\newcommand{\z}{{\bf z}}
\subsubsection*{Virial theorem}
The exact wave function obeys the \emph{virial theorem}, which states that (for a Coulombic potential $\hat V$) \cite{weissbluth}\comment{p570}
\begin{align}
\langle \hat T \rangle &= -\frac{1}{2}\langle \hat V\rangle. \label{eq:virial}
\end{align}
A simple proof\footnote{More precisely, a heuristic (not entirely rigorous) justification.} by dimensional analysis due to Weinberg \cite{weinberg}\comment{p190}{ }for the one-particle case illustrates the condtion: Since the square of the wave function has to integrate over space to a probability, it must have dimensions of $\text{Length}^{\nicefrac{-3}{2}}$. Letting $a$ denote the chosen length scale, we can express the wave function as $\psi(\r)=a^{\nicefrac{-3}{2}}f(\r/a)$, with $\z\equiv\r/a$ and $f(\z)$ being a dimensionless function of a dimensionless argument. Changing integration variables in the expressions for $\langle \hat V\rangle$ and $\langle \hat T\rangle$,
\begin{align}
\langle \psi|\hat V|\psi\rangle = \langle \hat V\rangle_\psi&= \frac{\int \mathrm{d}^3\r \, V(\r) |\psi(\r)|^2}{\int \mathrm{d}^3\r \,|\psi(\r)|^2}, \ \ \text{ and}\\
%
\langle \psi|\hat T|\psi\rangle = \langle \hat T \rangle_\psi &= \frac{\int \mathrm{d}^3\r \, \frac{\hbar^2}{2m}\left(|\pder{\psi}{x}|^2+|\pder{\psi}{y}|^2+|\pder{\psi}{z}|^2\right)}{\int \mathrm{d}^3\r \,|\psi(\r)|^2},
\end{align}
from $\r\rightarrow \z=\r/a$ gives a single factor of $a^{-1}$ in the former and $a^{-2}$ in the latter integral. For the denominators, the integration measure $\mathrm{d}^3\r$ carries dimensions of $a^3$ which exactly cancel the $(a^{\nicefrac{-3}{2}})^2=a^{-3}$ from the wave function squared (by neccessity, since the integral represents a probability). In the $\langle V\rangle$ integral, the same happens in the numerator, and we are left with only the Coulomb potential $a^{-1}$ contribution. The numerator in the $\langle T \rangle$ integral has dimensions of $a^{-2}$, since each coordinate differentiation carries a single inverse $a$.

As the exact wave function is a variational minimum of the Hamiltonian, the expectation value of $\langle H\rangle_\psi$ (note carefully that this is true only when evaluated \emph{at the exact wave function} [ground or excited states]) must be independent of variations in $\psi$. Namely, they must be independent of variations of $a$ since $\psi(\r)=a^{\nicefrac{-3}{2}}f(\z)$ \cite{weinberg}\comment{p190}. The derivative of $\langle \hat T\rangle_\psi + \langle \hat V \rangle_\psi$ taken at the exact wave function must vanish, giving \eq{virial}.

The virial theorem generalizes to $N$ particles in the same form as \eq{virial}.

\subsubsection{Cusp conditions}
In general, when charged particles approach each other the Coulombic $1/r$ term of the interaction energy diverges. In order for the energy to remain finite, the wave function needs to obey very specific sets of conditions dictating the behaviour of the discontinous derivatives at the collision points. First described by Kato \cite{kato}, such \emph{cusp conditions} describe known properties of the quantum system and wave function at the divergent points of the inter-electron and electron-nucleus Coulomb potentials. These two cases will be described in depth in sections \ref{section:eecusp} and \ref{section:encusp}. 



\subsection{Electron-nucleus cusp \label{section:encusp}}
%"This is valid in a non-relativistic treatment within the Born–Oppenheimer approximation, and assuming point-like nuclei." \url{https://en.wikipedia.org/wiki/Kato_theorem}
We will now consider in some detail the issue of the cusp condition arising from the singular Coulombic potential at the position of point-like nuclei. 

Let us consider a system of a single atom of charge $+Z$ with $N$ bound electrons. It will be useful in the following to define the \emph{local energy}, $E_\text{local}$, as a spatially dependent measure of the "instantaneous" energy of a system. We take
\begin{align}
E_\text{local}(\R) \equiv \frac{1}{\Psi(\R)}\hat H\Psi(\R)
\end{align}
to be the local energy, and note that for any eigenfunction $\Phi(\R)$ of $\hat H$, the local energy is constant for all configurations $\R=\{\r_1,\r_2,\dots,\r_N,\sigma_1,\sigma_2,\dots,\sigma_N\}$ ((Hjort-Jensen)). This is simply a trivial result of the Schrödinger equation, since $\hat H\Phi(\R)=E\Phi(\R)$ we find that
\begin{align}
E_\text{local}(\R)=\frac{1}{\Phi(\R)}\hat H\Phi(\R) = \frac{1}{\Phi(\R)}E\Phi(\R)=E. \label{eq:rads}
\end{align}
We cannot normally find an approximate wave function for which $E_\text{local}(\R)=E$ holds, but we should at least make sure that it is \emph{well behaved}. There are certain critical electronic configurations for which the Coulombic potential diverges. Keeping the local energy finite at these points lead to what is known as cusp conditions on the wave function. 

The first critical configuration we will consider is the class of electronic positions for which $|\r_i-\r_A|\rightarrow 0$ for some electron $i$ and nucleus $A$. Since the electron-nucleus Coulomb potential diverges there must be a corresponding divergent term in the laplacian which exactly cancels it. Let us consider the Born-Oppenheimer Hamiltonian for a single electron in the presence of a charge-$Z$ atom,
\begin{align}
\hat H(\r) &= -\frac{\nabla^2}{2} - \frac{Z}{|\r|},
\end{align}
where we take the atom to be situated at the origin. The \emph{radial} Schrödinger equation can be written as \cite{thijssen}\comment{p156}
\begin{align}
\left[\pder{^2}{r^2} + \frac{2}{r}\pder{}{r} + \frac{2Z}{r} - \frac{l(l+1)}{r^2} + 2E \right]R(r) = 0.
\end{align}
For $l=0$ states, we note that the two $1/r$ terms must exactly cancel if the local energy is to remain finite when $r\rightarrow 0$. This means that 
\begin{align}
E_\text{local}(\r\rightarrow0)=\lim_{r\rightarrow0} \left\{\frac{1}{R(r)}\left(\frac{2}{r}\pder{}{r} +\frac{2Z}{r} + \text{ finite terms} \right)R(r)\right\},
\end{align}
and the exact wave function obeys \cite{hammond}\comment{p156}
\begin{align}
\lim_{r\rightarrow0}\left\{ \frac{1}{R(r)} \pder{R}{r}\right\}=-Z.
\end{align}
We see that a wave function of s-type symmetry which does not vanish at $r=0$ must be exponential in $r$ in the limit of $r\rightarrow0$. 
\newcommand{\Rdo}{\pder{\tilde R}{r}}
\newcommand{\Rd}[1]{\pder{^#1\tilde R}{r^#1}}
What happens if $l\not=0$ or $R(0)=0$? It turns out that considering the latter issue automatically resolves the first, so let us take the case of $R(0)=0$ \cite{hammond}\comment{p157}. We may factor the leading $r$ dependence out of $R(r)$, and define $\tilde R(r)\equiv r^mR(r)$, so that $\tilde R(0)\not=0$. Three applications of the derivative product rule gives 
\begin{align}
\pder{}{r}R(r) &= \pder{}{r}\left[ \tilde R(r) r^m \right] = \Rdo r^m + m\tilde R(r)r^{m-1},
\end{align}
and
\begin{align}
\pder{^2}{r^2}R(r) &= \pder{^2}{r^2}\left[\tilde R(r) r^m\right] = \Rd{2} r^m + 2\Rdo mr^{m-1}+m(m-1)\tilde R(r)r^{m-2}.
\end{align}
Insertion into the radial Schrödinger equation, \eq{rads}, we find that
\begin{align}
& \Rd{2}r^m+\Rdo \frac{2(m+1)}{r}r^m + \tilde R(r)\frac{m(m+1)}{r^2}r^m + \nn\\
& \ \ \ \ \ \ \ \ \ \ \ \ \ \ \ \ \ \ \ \ \tilde R(r)\frac{2Z}{r}r^m-\tilde R(r) \frac{l(l+1)}{r^2}r^m + 2E\tilde R(r)r^m=0.
\end{align}
If the local energy is to remain finite once again, we need the inverse powers of $r$ to cancel, which for $1/r^2$ yields $m=l$. Furthermore, for the $1/r$ terms, we have the condition 
\begin{align}
\lim_{r\rightarrow0}\left\{ \frac{1}{\tilde R(r)} \pder{\tilde R}{r}\right\}=-\frac{Z}{l+1}.
\end{align}

%   The first critical configuration we will consider is the class of electronic positions for which $|\r_i-\r_A|\rightarrow 0$ for some electron $i$ and nucleus $A$. Since the electron-nucleus Coulomb potential diverges there must be a corresponding divergent term in the laplacian which exactly cancels it. Let us consider the Born-Oppenheimer Hamiltonian for a system of $N$ electrons in the presence of a single charge-$Z$ atom:
%   \begin{align}
%   \hat H(\R) &= \sum_{i=1}^N -\frac{\nabla^2_i}{2} + \sum_{i=1}^N\sum_{j=i+1}^N \frac{1}{|\r_i-\r_j|} - \sum_{i=1}^N \frac{Z}{|\r_i-\r_A|}.
%   \end{align}
%   Let us for simplicity take the position of nucleus, $\r_A$, to be the origin. We assume for the moment that all inter-electronic distances $r_{ij}$ are finite and non-zero. Writing out the radial Schrödinger equation and focusing on the terms originating from the first particle, we find
%   \begin{align}
%   \left[\pder{^2}{r^2_1} + \frac{2}{r_1}\pder{}{r_1} + \frac{2Z}{r_1^2} - \frac{l(l+1)}{r_1^2} + 2E + \sum_{i=2}^N F\left(r_i,\pder{}{r_i},\pder{^2}{r_i^2}\right)\right]R(r_1,r_2,\dots,r_N) = 0,
%   \end{align}
%   where $F$ is simply the remaining terms in the Hamiltonian containing inter-electronic interactions, electron-nucleus interactions, and the kinetic energy operators for electrons $i=2,3,\dots,N$.


\subsection{Electron-electron cusp \label{section:eecusp}}
In the limit of two colliding electrons $r_{12}\rightarrow 0$, another cusp condition is found. It turns out that the corresponding radial equation for the inter-electronic separation carries the same dependence on $r_{12}$ as the in the electron-nucleus case. With the only difference being the $-Z/r$ potential being replaced by a repulsive $1/r_{12}$ and kinetic term being twice as large \cite{thijssen}\comment{p380}. 

We can write down the divergent parts of the local energy as
\begin{align}
E_\text{local}(\r_{12}\rightarrow0)&=\lim_{r_{12}\rightarrow0} \left\{\frac{1}{R(r_{12})}\left(\frac{2}{r_{12}}\pder{}{r_{12}} - \frac{2}{r_{12}} -\right.\right.\nn\\
& \phantom{-----} \left.\left. \frac{l(l+1)}{r_{12}^2} + \text{ finite terms} \right)R(r_{12})\right\},
\end{align}
where $l=1$ if the spin-projections of electrons $1$ and $2$ are equal, and $l=0$ otherwise \cite{hjorth-jensen}\comment{p478}. A derivation analogue to the previous one reveals a corresponding cusp condition: 
\begin{align}
\lim_{r_{12}\rightarrow0}\left\{\pder{R}{r_{12}}\right\} = -\frac{R(r_{12})}{2(l+1)}.
\end{align}
This can be satisfied by a term in the wave function proportional to 
\begin{align}
R(r_{12})\propto \left\{ \mat{lcr}{
  \displaystyle\exp\left(\frac{r_{12}}{2}\right) & \text{ if } & \sigma_i=\sigma_j \\
  \\
  \displaystyle\exp\left(\frac{r_{12}}{4}\right) & \text{ if } & \sigma_i\not=\sigma_j
}\right..
\end{align}

\subsection{Higher order coalescence conditions}
In a system of $N+M$ charged particles, $N$ electrons and $M$ nuclei there will in general be a lot of such cusp conditions or \emph{coalescence points}, where two or more electrons or nuclei coalesce with each other. Assuming all nuclei have the same charge $Z$, and disregarding nucleus-nucleus coalescence (recall that the Born-Oppenheimer wave function depends only parametrically on the positions of the nuclei), leaves us with: $r_{ij}\rightarrow0$, $r_{ia}\rightarrow0$, $r_{ij}\rightarrow0$ and simulaneously $r_{ik}\rightarrow$, $r_{ia}\rightarrow0$ and simulaneously $r_{ak}\rightarrow$, etc. The $i,j,k,\dots$ indices here denote electronic coordinates, while $a,b,c,\dots$ denote nucleonic coordinates.

We will not consider higher order conditions in the present work, but refer the reader to e.g.\ \cite{hammond,assaraf}\comment{c5.3-5.4}\comment{c2}.

\section{Hydrogen wave functions}




\section{Jastrow factor}
Multiple functional forms which account explicitly for the electron-electron cusp condition described in section \ref{section:eecusp} are used in the literature. Some examples include the Boys-Handy function, the double exponential, or the Gaussian geminal form \cite{hammond}\comment{p173}. However, the most commonly used form is the Jastrow factor, sometimes called the Padé-Jastrow factor. Although originally proposed by Bijl in 1940, the form is commonly attributed to American physicist R. Jastrow \cite{anderson,bijl,jastrow}\comment{anderson p46}.

The two-body Jastrow factor used in the current work has the form
\begin{align}
J(\R)=\exp\left[\sum_{i=1}^N\sum_{j=i+1}^N \frac{a_{ij}r_{ij}}{1+\beta r_{ij}} \right],
\end{align}
where $\beta>0$ is a tunable parameter and $a$ depends on the relative spin-projections of electrons $i$ and $j$ as \cite{hjorth-jensen}
\begin{align}
a_{ij}=\left\{\mat{lcr}{
  \nicefrac{1}{4} & \text{ if } & \sigma_i=\sigma_j    \\
  \nicefrac{1}{2} & \text{ if } & \sigma_i\not=\sigma_j
} \right. .
\end{align}

In general, it is possible to add higher order polynomials terms to the exponent resulting in \cite{hammond}
\begin{align}
J(\R)=\exp\left[\sum_{i=1}^N\sum_{j=i+1}^N \frac{a_1r_{ij}+a_2r_{ij}^2+\dots}{1+\beta_1 r_{ij} + \beta_2 r_{ij}^2 + \dots} \right].
\end{align}
For optimized $\beta_k$ values this may yield a more precise approximation to the true wave function, but it comes at the cost of more difficult computation and parameter optimization.

\section{Orbitals}
As noted in section \ref{section:slater}, the Slater determinant\textemdash populated with eigenfunctions of the one-electron operator $\hat h$\textemdash is an exact solution for a non-interacting $N$-electron problem. But Slater determinants are also used as wave function ansatzes for interacting systems. The question of which functions should occupy the determinants however is a fairly non-trivial one. It turns out\textemdash somewhat surprisingly\textemdash that the chief concern is that of computational efficiency.

Before diving in, we detail briefly the \emph{spherical harmonics}.

\subsection{Spherical and solid harmonics}
The spherical harmonics are a set of functions defined on the surface of a sphere. They are complete in the sense that they span the space of complex-valued continous functions on the unit sphere, $\text{span}\{Y_l^m(\theta,\phi)\}=C(\mathbb{S})$, and the space of complex-valued square integrable functions on $\mathbb{S}$, $\text{span}\{Y^m_l(\theta,\phi)\}=L^2(\mathbb{S})$ \cite{atkinson}\comment{p8}. The surface of the unit sphere is here denoted by $\mathbb{S}=\{\r=(x,y,z):|\r|^2=1\}$. The spherical harmonics thus naturally arise as an expansion basis for functions defined on the sphere. 

Even more importantly, the spherical harmonics are eigenfunctions of the both the total angular momentum operator, $\hat {\bf L}^2$, and the $z$-projection of the angular momentum, $\hat L_z$ \cite{hassani}\comment{338}. For QM problems involving \emph{central potentials}\textemdash i.e.\ $V_\text{ext}=V_\text{ext}(r)$\textemdash the spherical harmonics are solutions to the angular part of the time independent Schrödinger equation (arising from separation of variables). 

Formally, the spherical harmonics are functions of $\theta$ and $\phi$: $Y^m_l(\theta,\phi)$ proprotional to $P^{|m|}_l(\theta)\mathrm{e}^{\mathrm{i}m\phi}$ (up to a normalization constant), where $P^{|m|}_l$ satisfies 
\begin{align}
-\frac{1}{\sin\theta}\der{}{\theta}\left( \sin\theta \der{P_l^{|m|}}{\theta} \right) + \frac{m^2}{\sin^2\theta}P^{|m|}_l = l(l+1)P_l^{|m|}.
\end{align}
The parameters $l$ and $m$ are both integers, with $l\ge0$ and $-l\ge m \ge l$. The polynomials $P^m_l$ are known as the associated Legendre functions. The combination $r^l Y_l^m$ is a homogenous polynomial\footnote{A homogenous polynomial is a polynomial in which all terms have the same degree, i.e.\ $x^2+xy+y^2$ or $x^3+y^3+z^3+x^2z$.} of order $l$ in the cartesian unit vector \cite{weinberg,atkinson}\comment{p37,p19}. These combinations are called the \emph{solid harmonics} (up to a normalization), and can be made real by taking linear combinations of $\pm m$ terms. The first few real solid harmonics, $S_l^m(x,y,z)$ are shown in \tab{solidharmonics}. 

\begin{table}[p]
\centering
\setlength\extrarowheight{2pt}
\begin{tabularx}{\textwidth}{l X  r}
\hline
\hline
\\[-0.9em]
{\bf Real solid}  & {\bf Azimuthal, magnetic}       & \\ 
{\bf harmonic}    & {\bf quantum numbers}, $(l,m)$  & {\bf Expression} \\
\\[-0.9em]
\hline
\\[-0.9em]
$S_0^0$    & \ \ \ \ \ \ \ \ \ \ \ \ 0, 0 & $\displaystyle 1$ \\
\\[-0.5em]

$S_1^{-1}$ & \ \ \ \ \ \ \ \ \ \ \ \ 1, -1 & $\displaystyle y$ \\
\\[-0.5em]
$S_1^0$    & \ \ \ \ \ \ \ \ \ \ \ \ 1, 0 & $\displaystyle z$ \\
\\[-0.5em]
$S_1^1$    & \ \ \ \ \ \ \ \ \ \ \ \ 1, 1 & $\displaystyle x$ \\
\\[-0.5em]

$S_2^{-2}$    & \ \ \ \ \ \ \ \ \ \ \ \ 2, -2 & $\displaystyle \sqrt{3}xy$ \\
\\[-0.5em]
$S_2^1$       & \ \ \ \ \ \ \ \ \ \ \ \ 2, 1 & $\displaystyle \sqrt{3}yz$ \\
\\[-0.5em]
$S_2^0$       & \ \ \ \ \ \ \ \ \ \ \ \ 2, 0 & $\displaystyle -\frac{1}{2}\left(x^2+y^2\right)+z^2$ \\
\\[-0.5em]
$S_2^{-1}$    & \ \ \ \ \ \ \ \ \ \ \ \ 2, 1 & $\displaystyle \sqrt{3}xz$ \\
\\[-0.5em]
$S_2^{-2}$    & \ \ \ \ \ \ \ \ \ \ \ \ 2, 2 & $\displaystyle \frac{\sqrt{3}}{2}(x-y)(x+y)$ \\
\\[-0.5em]

$S_3^{-3}$    & \ \ \ \ \ \ \ \ \ \ \ \ 3, -3 & $\displaystyle -\frac{1}{2}\sqrt{\frac{5}{2}}y(-3x^2+y^2)$ \\
\\[-0.5em]
$S_3^2$       & \ \ \ \ \ \ \ \ \ \ \ \ 3, -2 & $\displaystyle \sqrt{15}xyz$ \\
\\[-0.5em]
$S_3^{1}$     & \ \ \ \ \ \ \ \ \ \ \ \ 3, -1 & $\displaystyle -\frac{1}{2}\sqrt{\frac{3}{2}}y(x^2+y^2-4z^2)$ \\
\\[-0.5em]
$S_3^0$       & \ \ \ \ \ \ \ \ \ \ \ \ 3, 0 & $\displaystyle -\frac{3}{2}\left(x^2+y^2\right)z+z^3$ \\
\\[-0.5em]
$S_3^{-1}$    & \ \ \ \ \ \ \ \ \ \ \ \ 3, 1 & $\displaystyle -\frac{1}{2}\sqrt{\frac{3}{2}}x(x^2+y^2-4z^2)$ \\
\\[-0.5em]
$S_3^{-2}$    & \ \ \ \ \ \ \ \ \ \ \ \ 3, 2 & $\displaystyle \frac{1}{2}\sqrt{15}(x-y)(x+y)z$ \\
\\[-0.5em]
$S_3^{-3}$    & \ \ \ \ \ \ \ \ \ \ \ \ 3, 3 & $\displaystyle \frac{1}{2}\frac{\sqrt{5}}{2}x(x^2-3y^2)$ \\
\\[-0.5em]
\hline
\end{tabularx}
\caption{Examples of the first few \emph{real solid harmonics}, $S^m_l(x,y,z)=r^lY^m_l(\theta,\phi)$. We note that the solid harmonics of order $l$ are simply linearly independent homogenous polynomials in $x$, $y$, and $z$, of order $l$.  \label{tab:solidharmonics}}
\end{table}

The real solid harmonics are often a more convenient form to work with, simply because they are real and defined in terms of cartesian coordinates. Please note that the solid harmonics span $L^2(\mathbb{S})$ in the same way the spherical harmonics do, so we can do this without any loss of generality or applicability.

\subsection{Hydrogenic orbitals}
The non-interacting hydrogen-like Hamiltonian 
\begin{align}
\hat H = -\frac{\nabla^2}{2} - \frac{Z}{|\r-\r_A|},
\end{align}
with $\r_A$ being the position of a single nucleus of charge $+Z$, has normalized \emph{radial} eigenfunctions 
\begin{align}
R_{nl}(r)=\sqrt{\left(\frac{2Z}{n}\right)^3 \frac{(n-l-1)!}{2n[(n+l)!]^3}} \exp\left[-\frac{Zr}{n}\right]\left(\frac{2Zr}{n}\right)^l L_{n-l-1}^{2l+1}\left(\frac{2Zr}{n}\right). \label{eq:hydrogenanalytic}
\end{align}
The $L$ here denotes the (generalized) Laguerre polynomials\footnote{The (generalized) Laguerre polynomials are the solutions to the differential equation
\begin{align}
x\der{^2y(x)}{x^2}+\left(1+\alpha-x\right)\der{y(x)}{x}+ny(x)=0,
\end{align}
with $n$ a non-negative integer and $\alpha$ an arbitrary real constant \cite{rottmann}. An explicit expression for the polynomials themselves can be found by the so-called Rodrigues formula:
\begin{align}
L_n^\alpha(x) = x^{-\alpha}\frac{1}{n!}\left(\der{}{x}-1\right)^nx^{n+\alpha}.
\end{align}} In order to produce the full spin-orbitals, we need to append the spherical harmonic $Y_l^m(\theta,\phi)$ for appropriate quantum numbers $l$ (the azimuthal quantum number) and $m$ (the magnetic quantum number) in addition to a spin function, $\chi(\sigma)$. The corresponding eigenvalues depend famously only on the principal quantum number $n$, as \cite{griffiths}\comment{149}
\begin{align}
E=-\frac{Z^2}{2n^2}. 
\end{align} 
The first six radial orbitals are shown in \fig{hydrogenorbital}, with explicit expressions for the first ten being shown in \tab{hydrogenorbitals}.
\fig{hydrogenorbital3d} shows a few examples of the full orbitals, i.e.\ the radial functions multiplied by spherical harmonics.

\begin{table}
\centering
\setlength\extrarowheight{2pt}
\begin{tabularx}{\textwidth}{l X  r}
\hline
\hline
\\[-0.9em]
{\bf Radial}  & {\bf Principal, azimuthal}     & \\ 
{\bf orbital} & {\bf quantum numbers}, $(n,l)$ & {\bf Expression} \\
\\[-0.9em]
\hline
\\[-0.9em]
$R_{10}$ & \ \ \ \ \ \ \ \ \ \ \ \ 1, 0 & $\displaystyle 2\sqrt{Z^3}\mathrm{e}^{-Zr}$ \\
\\[-0.5em]
$R_{20}$ & \ \ \ \ \ \ \ \ \ \ \ \ 2, 0 & $\displaystyle \sqrt{\frac{Z^3}{2}}\left(1-\frac{Zr}{2}\right)\mathrm{e}^{-Zr/2}$ \\
\\[-0.5em]
$R_{21}$ & \ \ \ \ \ \ \ \ \ \ \ \ 2, 1 & $\displaystyle \sqrt{\frac{Z^5}{24}} r \mathrm{e}^{-Zr/2}$ \\
\\[-0.5em]
$R_{30}$ & \ \ \ \ \ \ \ \ \ \ \ \ 3, 0 & $\displaystyle \frac{2\sqrt{Z^3}}{\sqrt{27}} \left( 1-\frac{2Zr}{3}+\frac{2Z^2r^2}{27}\right) \mathrm{e}^{-Zr/3}$ \\
\\[-0.5em]
$R_{31}$ & \ \ \ \ \ \ \ \ \ \ \ \ 3, 1 & $\displaystyle \frac{8\sqrt{Z^5}}{27\sqrt{6}} \left( 1-\frac{Zr}{6}\right)r \mathrm{e}^{-Zr/3}$ \\
\\[-0.5em]
$R_{32}$ & \ \ \ \ \ \ \ \ \ \ \ \ 3, 2 & $\displaystyle \frac{4\sqrt{Z^7}}{81\sqrt{30}} r^2 \mathrm{e}^{-Zr/3}$ \\
\\[-0.5em]
$R_{40}$ & \ \ \ \ \ \ \ \ \ \ \ \ 4, 0 & $\displaystyle \frac{\sqrt{Z^3}}{4} \left(1-\frac{3Zr}{4} + \frac{Z^2r^2}{8}-\frac{Z^3r^3}{192}\right)  \mathrm{e}^{-Zr/4}$ \\
\\[-0.5em]
$R_{41}$ & \ \ \ \ \ \ \ \ \ \ \ \ 4, 1 & $\displaystyle \frac{\sqrt{5Z^5}}{16\sqrt{3}} \left(1-\frac{Zr}{4} + \frac{Z^2r^2}{80}\right)r  \mathrm{e}^{-Zr/4}$ \\
\\[-0.5em]
$R_{42}$ & \ \ \ \ \ \ \ \ \ \ \ \ 4, 2 & $\displaystyle \frac{\sqrt{Z^7}}{64\sqrt{5}} \left(1-\frac{Zr}{12}\right)r^2  \mathrm{e}^{-Zr/4}$ \\
\\[-0.5em]
$R_{43}$ & \ \ \ \ \ \ \ \ \ \ \ \ 4, 3 & $\displaystyle \frac{\sqrt{Z^9}}{768\sqrt{35}} r^3  \mathrm{e}^{-Zr/4}$ \\
\\[-0.5em]
\hline
\end{tabularx}
\caption{Explicit analytical expressions for the first few hydrogenic radial wave functions, $R_{nl}(r)$ \cite{griffiths}\comment{p154}. The first six\textemdash $R_{10}$ to $R_{32}$\textemdash are shown for $Z=1$ in \fig{hydrogenorbital}. \label{tab:hydrogenorbitals}}
\end{table}

\begin{figure}[p!]
\centering
\includegraphics[width=0.49\textwidth]{hydrogenOrbitals1.pdf}
\includegraphics[width=0.49\textwidth]{hydrogenOrbitals2.pdf}
\vspace{-60pt}
\caption{The first six hydrogenic orbitals for $Z=1$, $R_{nl}(r)$. The general expression for $R_{nl}(r)$ is given in \eq{hydrogenanalytic}. Explicit expressions for these orbitals are given in \tab{hydrogenorbitals}. \label{fig:hydrogenorbital}}
\end{figure}

\begin{figure}[p!]
\centering
\includegraphics[width=0.49\textwidth,trim=250 250 250 250, clip]{H100.png}
\includegraphics[width=0.49\textwidth,trim=250 250 250 250, clip]{H210.png}
\includegraphics[width=0.49\textwidth,trim=150 150 150 150, clip]{H311.png}
\includegraphics[width=0.49\textwidth,trim=250 250 250 250, clip]{H420.png}
\caption{Examples of full hydrogenic orbitals, $\psi_{nlm}(r,\theta,\phi)=R_{nl}(r)Y^m_l(\theta,\phi)$: $\psi_{100}$ (top left), $\psi_{210}$ (top right), $\psi_{311}$ (bottom left), and $\psi_{420}$ (bottom right). The relative scaling is not accurate. Each plot is sliced in the $x$-$y$-plane and a colormap showing the density is inset. The outer contour shows the isosurface of each orbital at $|\psi_{nlm}|^2=10^{-5}$. \label{fig:hydrogenorbitals3d}}
\end{figure}

In many ways, the hydrogenic orbitals are \emph{natural} orbitals to work with. They obey\textemdash crucially\textemdash the electron-nucleus cusp condition of section \ref{section:encusp}. They also fall of exponentially as per the correct long range limit. However, integral evaluation with the hydrogenic orbitals turns out to be unfeasibly slow compared to more efficient basis sets. Ultimately, we will end up working with a basis set intrinsicly \emph{less suited} to the task because the neccessary integrals are possible to evaluate quickly. We expand on this further in section \ref{section:gaussianorbitals} 

\subsection{Slater type orbitals}
A relative of the hydrogenic orbitals are the Slater type orbitals (STO). They have the same exponential structure, 

The general expression for the $(n,l,m)$ STO with exponent $\zeta$ is given by \cite{cramer}\comment{p134}
\begin{align}
\psi_{nlm}(r,\theta,\phi;\zeta)=\frac{(2\zeta)^{n+\nicefrac{1}{2}}}{[(2n)!]^{\nicefrac{1}{2}}} r^{n-1}\mathrm{e}^{-\zeta r} Y^m_l(\theta,\phi).
\end{align}

\subsection{Gaussian type orbitals \label{section:gaussianorbitals}}


\begin{figure}
\centering
\includegraphics[width=0.49\textwidth,trim=250 250 250 250, clip]{G000.png}
\includegraphics[width=0.49\textwidth,trim=250 250 250 250, clip]{G100.png}
\includegraphics[width=0.49\textwidth,trim=250 250 250 250, clip]{G201.png}
\includegraphics[width=0.49\textwidth,trim=250 250 250 250, clip]{G002m020m200.png}
\caption{Examples of Gaussian orbitals with $\alpha=1.0$, $G_{ijk}(x,y,z)=x^iy^jz^k\mathrm{e}^{-\alpha r^2}$: $G_{000}$ (top left), $G_{100}$ (top right), $G_{201}$ (bottom left), and $G_{002}-G_{020}-G_{200}$ (bottom right). The latter combination is one of the five linearly independent set of d-type Gaussian orbitals (there are a total of six Gaussian primitives with $l=2$\textemdash $G_{200}$, $G_{020}$, $G_{002}$, $G_{110}$, $G_{101}$, and $G_{011}$\textemdash but the set is linearly dependent: a linearly independent set may be formed by taking $G_{110}$, $G_{101}$, $G_{011}$, $G_{200}-G_{020}$, and $G_{002}-G_{020}-G_{200}$). Each plot is sliced in the $x$-$y$-plane and a colormap showing the density is inset. The outer contour shows the isosurface of each orbital at $|G_{ijk}|^2=10^{-5}$. \label{fig:gaussianorbitals3d}}
\end{figure}




















\end{document}

% \begin{figure}[p!]
% \centering
% \includegraphics[width=12cm]{<fig>.pdf}
% \caption{\label{fig:1}}
% \end{figure}
 
% \lstinputlisting[firstline=1,lastline=2, float=p!, caption={}, label=lst:1]{<code>.m}

