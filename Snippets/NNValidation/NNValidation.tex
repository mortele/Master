\documentclass[../../master.tex]{subfiles}

\begin{document}

\chapter{Neural Network validation tests \label{NNvalidation}}
\section{Single variable curve fit}
The natural place to start the testing of our Neural Network potential fit scheme is with a simple function of a single variable, $f:\mathbb{R}\rightarrow\mathbb{R}$. As an initial test, the specific functional form is of little importance. However\textemdash in the spirit of the present contex\textemdash we choose a Lennard-Jones (LJ) parameterization,
\begin{align}
V_\text{LJ}(r) = \frac{1}{r^{6}}-\frac{1}{r^{12}}.
\end{align}
For the moment we forget about the normalization, and the scaling of the distances by the usual $\sigma$ parameter. Using a simple network structure of a single hidden layer consisting of 10 neurons, we train for $10^3$ with a data set consisting of $10^6$ samples of the LJ potential for $0.9\le r\le 1.6$. The resulting network output and training details are shown in \fig{ljtrain}. We note the network output and the training data coincide\textemdash approximately\textemdash perfectly after 1000 epochs of training, with the average squared difference between the validation points and the true potential is on the order of $~10^{-7}$. 

From the bottom graph of \fig{ljtrain}, it is clear that an overall minimum has not yet been reached: the cost is still steadily decreasing (albeit slowly, note the logarithmic $y$-axis scaling). 

\begin{figure}
\centering
\includegraphics[width=0.49\textwidth,trim=0  200 0 200, clip]{ljtrain1.pdf}
\includegraphics[width=0.49\textwidth,trim=0  200 0 200, clip]{ljtrain2.pdf}
\includegraphics[width=0.49\textwidth,trim=0  200 0 200, clip]{ljtrain3.pdf}
\includegraphics[width=0.49\textwidth,trim=0  200 0 200, clip]{ljtrain4.pdf}
\includegraphics[width=0.79\textwidth,trim=0  200 0 200, clip]{ljtrain.pdf}
\caption{Samples showing the training of an ANN on samplings of the Lennard-Jones potential between $0.9\le r \le 1.6$. The evolution of the network output with increasing epochs is shown (top) in addition to the cost as a function of epoch (bottom).  \label{fig:ljtrain}}
\end{figure}


\section{Approximating noisy data \label{noisesec}}
When performing \emph{ab initio} calculations using VMC the results will always be slightly distorted due to the statistical nature of the method. The statistical errors can be made arbitrarily small by increasing the sampling set, but for a finite number of Monte Carlo samples, it will never be identically zero\footnote{Assuming for the moment that the \emph{true} wave function is not known, in which case any number of cycles would give zero statistical error.}. Because of this, it is inherently essential that our ANN potential is able to perform it's job in the presence of noisy data. 

Thus, in order to test the suitability of our model in the presence of noise, we generate a sine curve with Gaussian noise. Since we ideally want the NN to be able to handle noise of a wide range of frequencies, we add the random fluctuations of different frequencies. In order to easily generate such a data set, we start from the Fourier coefficients of the data. We want the data to not be \emph{dominated} by noise, so we set $a_2=1$, and then take
\begin{align}
a_k\propto N\left[0,\,\frac{1}{2k}\right], \ \ \text{ for } \ \ k=30,31,\dots,99,100.
\end{align}
The $N(\mu,\sigma)$ denotes a random Gaussian of mean $\mu$ and standard deviation $\sigma$. Taking the \emph{real} inverse Fourier transform of the $\bm{\alpha}$ vector, $f(x)\equiv\mathcal{F}^{-1}[\bm{\alpha}]$ now yields a sine curve with random noise of differing frequencies. An exmaple of such a data set is shown in \fig{noise1}.  
\begin{SCfigure}
\centering
\includegraphics[width=0.49\textwidth,trim=0  200 0 200, clip]{noise1.pdf}
\caption{Example data set created to contain noise of different frequencies, as described in section \ref{noisesec}. The red curve is the one used for the training described in the same section.\label{fig:noise1}}
\end{SCfigure}

For the training we use a neural network consisting of a single layer of 20 neutrons. A selection of training snapshots are shown in \fig{noise2}. Also shown is the magnitude of the cost function plotted versus the epoch number. In order to validate the training, we use a separate set of data points\textemdash which the NN is never trained on\textemdash to check the state of the training. Since we are interested in a network solution independent of the noise, only capturing the underlying shape, we use a validation set with the same structure but with \emph{different} noise. For simplicity, we use the data set shown in red in \fig{noise1}. We note that the cost functions relative to both data sets fall off in mostly the same fashion, with an apparent slight difference in the fully trained state. 

Crucially, we see no signs of \emph{over-training}, which would cause the NN output to begin to follow the structure of the training data \emph{noise}. Over-training can be seen\textemdash if it happens\textemdash from the contiuing fall of the cost function relative to the training data, with a simultaneous \emph{increase} in the cost function relative to the validation set. This is not seen in our model. We note that apparently the randomization of the order of the training input, aswell as the online learning scheme efficiently counteracts the over-training phenomenon in the current model.
\begin{figure}
\centering
\includegraphics[width=0.49\textwidth,trim=0  200 0 200, clip]{noisetrain11.pdf}
\includegraphics[width=0.49\textwidth,trim=0  200 0 200, clip]{noisetrain22.pdf}
\includegraphics[width=0.49\textwidth,trim=0  200 0 200, clip]{noisetrain33.pdf}
\includegraphics[width=0.49\textwidth,trim=0  200 0 200, clip]{noisetrain44.pdf}
\includegraphics[width=0.49\textwidth,trim=0  200 0 200, clip]{noisetrain55.pdf}
\includegraphics[width=0.49\textwidth,trim=0  200 0 200, clip]{noisetrain66.pdf}
\caption{Training snapshots for the noisy data test of section \ref{noisesec}. The training set is shown in blue, while the NN output is shown in red. The epoch number\textemdash denoting the length of the training\textemdash is inset for each graph. The bottom right plot shows the evolution of the cost function\textemdash$\Vert {\bf y}-\hat {\bf y}\Vert_2$\textemdash as a function of the epoch number for the training data and the separate validation set.\label{fig:noise2}}
\end{figure}

\begin{SCfigure}
\centering
\includegraphics[width=0.49\textwidth,trim=0  200 0 200, clip]{sw.pdf}
\caption{The radial part of the SW molecular dynamics potential, used as an example data set for multi-variable potential fitting. The set contains combinations of $r_1$ and $r_2$ values for $0.8<r<a=1.8$. \label{fig:sw}}
\end{SCfigure}
\renewcommand{\r}{{\bf r}}
\section{Multi-variable fitting}
So far we have tried single-variable functions only. In the following, we explore the performance of our ANN model as a multi-variable curve fitting tool. We choose\textemdash entirely arbitrarily\textemdash a functional form based on the Stillinger-Weber (SW) molecular dynamics potential, originally developed to model \ce{Si} interacting atoms \cite{stillinger}. We consider configurations of three \ce{Si} atoms, indexed by $i$, $j$, and $k$. For simplicity, we use only the radial component, holding the $\theta_{ijk}$ angle constant at $\pi/2$. The radial part of the SW potential (in units of the $\varepsilon$ parameter) takes the form between 
\begin{align}
V(\r_i,\r_j,\r_k) = \phi(r_{ij}) + \phi(r_{ik}) + \phi(r_{jk}),
\end{align} 
with 
\begin{align}
\phi(r)\equiv \left\{
\mat{lcc}{
  \displaystyle A \left(\frac{B}{r^4}-\frac{1}{r}\right)\exp\left[\frac{1}{r-a}\right] & \text{ for } & r<a \\ \\
  \displaystyle 0 & \text{ for } & r\ge a
}\right.
\end{align}
We use the parameters $A=7.049556277$ and $B=0.6022245584$ according to Stiller and Webers original suggestions, with the distance scaling $a=1.8$. 

The training data set is shown in \fig{sw}. We use a grid of $r_{ij}$ and $r_{ik}$ values with $0.8\le r\le a$. For the training, we set up a NN with three layers of 20 neurons each. We allow the training to run a set of $10^6$ samples of the potential $V(r_1,r_2)$ for a total of $10^4$ epochs. The training is visualized in \fig{swtrain}, from which we note that the resulting network output is indistinguishable from the training set of \fig{sw}. From the cost evolution we conclude that the ANN model we implement is able to also approximate multi-dimensional data sets satisfactorily.  

\begin{figure}
\centering
\includegraphics[width=0.99\textwidth,trim=0 0 0 0, clip]{swtrain1.png}
\includegraphics[width=0.49\textwidth,trim=0 0 0 0, clip]{swtrain2.png}
\includegraphics[width=0.49\textwidth,trim=0  200 0 200, clip]{swtrainerror.pdf}
\caption{Example network trained to approximate the radial SW potential shown in \fig{sw}. The absolute error w.r.t.\ the training data (top) and the network output post training (bottom left) is shown, as well as the evolution of the cost as a function of epoch number (bottom right). We note that the functional form of the output from the network is indistinguishable from the training set shown in \fig{sw}. The approximation error is on the order of $\sim0.005\varepsilon$ for the interior points, but increases \emph{some} towards the edges of the training set. Note that the validation cost closely follows the training cost, indicating no over-training is happening.\label{fig:swtrain}}
\end{figure}

\section{Training on \emph{ab initio} data}
Out first foray into the use of \emph{ab initio} data will be a case study of the required amount of data. Since QM calculations are exceedingly expensive from a computational perspective, it is interesting to explore how many calculations we \emph{need} to perform in order to have enough data to train a ANN adequately. For this test we employ the \ce{H2+} data set\textemdash  presented in section \ref{hfvalid}\textemdash at the 6-311++G** level. 

In order to explore the training behaviour of the NN on small data sets, we train different networks using only parts of the complete data set, but retain\textemdash crucially\textemdash the complete set for validation. In \fig{smallsets} we present data from training runs using $N=50,75,100,\dots,475,500$ total data points. The removal of training pairs is done by excluding a given number of data points at random:
\begin{lstlisting}[
language=Python,
classoffset=4, % CLASSES
morekeywords={NeuralNetwork,initializeWeight,layer,initializeBias,fullNetwork, parseTypeString,constructNetwork,train,trainNetwork},
keywordstyle=\color{listingsclasscolor},
classoffset=5, % MEMBER VARIABLES
morekeywords={self},
keywordstyle=\color{listingsmembercolor},
classoffset=6, % KEY WORDS
morekeywords={__call__,__init__},
keywordstyle=\color{listingskeywordcolor},
classoffset=0,
morekeywords={with,None}]
xData = np.asarray(inputFileData).reshape([dataSize, self.inputs])
if N < dataSize :
		# If the requested size is smaller than the total data size, we prune
		# the set at random.
		toRemove = dataSize - N
		toRemove = np.random.choice.(	np.arange(dataSize), 
																	toRemove, 
																	replace=False)
		xData = np.delete(xData, toRemove)
\end{lstlisting}
This ensures we are still training on a representative subset of the full data set. 

\begin{figure}
\centering
\includegraphics[width=0.49\textwidth,trim=0  200 0 200, clip]{sizetest1.pdf}
\includegraphics[width=0.49\textwidth,trim=0  200 0 200, clip]{sizetest2.pdf}
\includegraphics[width=0.49\textwidth,trim=0  200 0 200, clip]{sizecost.pdf}
\caption{Overview of the effect of small sample sizes on the ANN training process. \label{fig:sizes}}
\end{figure}


\end{document} 



% \begin{figure}[p!]
% \centering
% \includegraphics[width=12cm]{<fig>.pdf}
% \caption{\label{fig:1}}
% \end{figure}