\documentclass[twoside,english]{uiofysmaster}
% STUFF IS LOCATED IN /Users/morten/.texmf/tex/latex/uiofysmaster
%
% Check 
% 
% https://tex.stackexchange.com/questions/96976/install-custom-cls-using-tex-live-in-local-directory
%
% for how to fix placement.

%Included packages ----------------------------------------------------------%
\usepackage[utf8]{inputenc}                  % utf-8 encoding, æ, ø , å, etc.
\usepackage{subfiles}
%\usepackage{a4wide}                          % Adjust margins to better fit A4 format.
\usepackage{array}                           % Matrices.
\usepackage{amsmath}                         % Math symbols, and enhanced matrices.o
\usepackage{amsfonts}                        % Math fonts.
\usepackage{amssymb}                         % Additional symbols.
%\usepackage{wasysym}                        % More additional symbols.
\usepackage{mathrsfs}                        % Most additional symbols.
%\usepackage[pdftex]{graphicx}                % Improved inclusion of .pdf-graphics files.
\usepackage[outercaption,wide]{sidecap}      % Floats with captions to the right/left.
\usepackage{cancel}                          % Visualize cancellations in equations.
\usepackage{enumerate}                       % Change counters (arabic, roman, etc.).
\usepackage{units}                           % Adds better looking fractions (nicefrac).
\usepackage{floatrow}                        % Multi-figure floats.
\usepackage{subfig}                          % Multi-figure floats.
\usepackage[margin=1cm]{caption}             % Adds functionality to captions.
\usepackage{bm}                              % Bolded text in math mode.
\usepackage{combinedgraphics}                % Figures; let latex handle the text itself.
\usepackage[framemethod=default]{mdframed}   % Make boxes.
\usepackage{listings}                        % For including source code.
\usepackage{soul}                            % Make vertical bars through text.
\usepackage{nicefrac}                        % Nice fractions with \nicefrac.
\usepackage{mathtools}                       % Underbrackets, overbrackets.
\usepackage{wasysym}                         % \smiley{}-s!
\usepackage{multicol}                        % Multiple text columns.
\usepackage{capt-of}                         % Caption things which are not floats.
%\usepackage[url=false]{biblatex}            % Citations (made easy).
\usepackage{dsfont}
\usepackage{booktabs}                        % Tables
\usepackage{tabularx}
\usepackage{array}
\usepackage{multirow}% http://ctan.org/pkg/multirow
\usepackage{hhline}% http://ctan.org/pkg/hhline
\usepackage{siunitx}
\usepackage[version=4]{mhchem}
\usepackage{relsize}  % Resize parts of equations
\usepackage[backend=biber,sorting=none]{biblatex}
\usepackage[version=4]{mhchem}
\usepackage{ifxetex,ifluatex}
\usepackage{etoolbox}
\usepackage{tikz}
\usepackage{framed}
\usepackage{relsize}
\usetikzlibrary{matrix}

\captionsetup{labelfont=bf}

% Quotes --------------------------------------------------------------------- %
% conditional for xetex or luatex
\newif\ifxetexorluatex
\ifxetex
  \xetexorluatextrue
\else
  \ifluatex
    \xetexorluatextrue
  \else
    \xetexorluatexfalse
  \fi
\fi
%
\ifxetexorluatex%
  \usepackage{fontspec}
  \usepackage{libertine} % or use \setmainfont to choose any font on your system
  \newfontfamily\quotefont[Ligatures=TeX]{Linux Libertine O} % selects Libertine as the quote font
\else
  \usepackage[utf8]{inputenc}
  \usepackage[T1]{fontenc}
  \usepackage{libertine} % or any other font package
  \newcommand*\quotefont{\fontfamily{LinuxLibertineT-LF}} % selects Libertine as the quote font
\fi

\newcommand*\quotesize{60} % if quote size changes, need a way to make shifts relative
% Make commands for the quotes
\newcommand*{\openquote}
   {\tikz[remember picture,overlay,xshift=-4ex,yshift=-2.5ex]
   \node (OQ) {\quotefont\fontsize{\quotesize}{\quotesize}\selectfont``};\kern0pt}

\newcommand*{\closequote}[1]
  {\tikz[remember picture,overlay,xshift=4ex,yshift={#1}]
   \node (CQ) {\quotefont\fontsize{\quotesize}{\quotesize}\selectfont''};}

% select a colour for the shading
\colorlet{shadecolor}{listingsbackgroundcolor}

\newcommand*\shadedauthorformat{\emph} % define format for the author argument

% Now a command to allow left, right and centre alignment of the author
\newcommand*\authoralign[1]{%
  \if#1l
    \def\authorfill{}\def\quotefill{\hfill}
  \else
    \if#1r
      \def\authorfill{\hfill}\def\quotefill{}
    \else
      \if#1c
        \gdef\authorfill{\hfill}\def\quotefill{\hfill}
      \else\typeout{Invalid option}
      \fi
    \fi
  \fi}
% wrap everything in its own environment which takes one argument (author) and one optional argument
% specifying the alignment [l, r or c]
%
\newenvironment{shadequote}[2][l]%
{\authoralign{#1}
\ifblank{#2}
   {\def\shadequoteauthor{}\def\yshift{-2ex}\def\quotefill{\hfill}}
   {\def\shadequoteauthor{\par\authorfill\shadedauthorformat{#2}}\def\yshift{2ex}}
\begin{snugshade}\begin{quote}\openquote}
{\shadequoteauthor\quotefill\closequote{\yshift}\end{quote}\end{snugshade}}


% Differentials -------------------------------------------------------------- %
\newcommand{\dt}{\,\mathrm{d}t}
\newcommand{\dx}{\,\mathrm{d}x}
\newcommand{\dr}{\,\mathrm{d}r}

% Derivatives ---------------------------------------------------------------- %
\newcommand{\der} [2]{\frac{\mathrm{d} #1}{\mathrm{d} #2}}   % Derivative.
\newcommand{\pder}[2]{\frac{\partial   #1}{\partial   #2}}   % Partial derivative.

% Matrices ------------------------------------------------------------------- %
\newcommand{\mat} [2]{\begin{matrix}[#1]  #2 \end{matrix}}   % Nothing enclosing it.
\newcommand{\pmat}[2]{\begin{pmatrix}[#1] #2 \end{pmatrix}}  % Enclosing parentheses.
\newcommand{\bmat}[2]{\begin{bmatrix}[#1] #2 \end{bmatrix}}  % Enclosing square brackets.
\newcommand{\vmat}[2]{\begin{vmatrix}[#1] #2 \end{vmatrix}}  % Enclosing vertical bars.
\newcommand{\Vmat}[2]{\begin{Vmatrix}[#1] #2 \end{Vmatrix}}  % Enclosing double bars.

% Number sets ---------------------------------------------------------------- %
\newcommand{\R}{\mathbb{R}}
\newcommand{\Q}{\mathbb{Q}}
\newcommand{\N}{\mathbb{N}}
\newcommand{\Z}{\mathbb{Z}}
\newcommand{\C}{\mathbb{C}}

% Manually set alignment of rows / columns in matrices (mat, pmat, etc.) ----- %
\makeatletter
\renewcommand*\env@matrix[1][*\c@MaxMatrixCols c]{%
  \hskip -\arraycolsep
  \let\@ifnextchar\new@ifnextchar
  \array{#1}}
\makeatother

% References ----------------------------------------------------------------- %
\newcommand{\Fig}[1]{Fig.\ \ref{fig:#1}}
\newcommand{\fig}[1]{Fig.\ \ref{fig:#1}}
\newcommand{\eq} [1]{Eq.\ (\ref{eq:#1})}
\newcommand{\Eq} [1]{Eq.\ (\ref{eq:#1})}
\newcommand{\tab}[1]{Table \ref{tab:#1}}
\newcommand{\Tab}[1]{Table \ref{tab:#1}}

% Paragraph formatting ------------------------------------------------------- %
\setlength{\parindent}{5.5mm}
\setlength{\parskip}  {0mm}

% Code listings font --------------------------------------------------------- %
\lstset{basicstyle=\scriptsize}
\usepackage{ifxetex}
\ifxetex
  \usepackage{fontspec}
  \newfontfamily\listingsfontfamily[Scale=0.85]{Droid Sans Mono}
  \renewcommand{\listingsfont}{\listingsfontfamily}
\fi

\newcommand{\inlinecc}[1]{\lstinline[language={[std]c++}]{#1}}


% Convenient shorthand notation ---------------------------------------------- %
\newcommand{\nn}{\nonumber}
\newcommand{\e}[1]{\cdot10^{#1}}
\renewcommand{\i}{\hat{\imath}}
\renewcommand{\j}{\hat{\jmath}}
\renewcommand{\k}{\hat{k}}
\usepackage{ifthen}
\newcommand*\CC{\texttt{C}\kern-0.2ex\raisebox{0.2ex}{\scalebox{0.9}{+\kern-0.2ex+}}}

% Caption position of tables at the top -------------------------------------- %
\floatsetup[table]{capposition=top}

% Black frame with white background ------------------------------------------ %
\newmdenv[linecolor=black,backgroundcolor=white]{exframe}
\newmdenv[linecolor=white,backgroundcolor=shadecolor]{shadeframe}

% Including vector drawings from inkscape ------------------------------------ %
\newenvironment{combFig}[5]{
  \begin{figure}[#1] 
    \centering 
    \includecombinedgraphics[vecscale=#2, keepaspectratio]{#3} 
    \caption{#4 \label{#5}}
  \end{figure}
  }

  {
}

% Including pdf graphics ----------------------------------------------------- %
\newenvironment{pdfFig}[5]{
  \begin{figure}[#1] 
    \centering 
    \includegraphics[width= #2]{#3} 
    \caption{#4 \label{#5}}
  \end{figure}
  }

  {
}


% Set bibliography file and path for images.
\graphicspath{{./images/}}
\newcommand{\includepdfgraphics}[2]{\includecombinedgraphics[#1]{./images/#2}}


\graphicspath{{/Users/morten/Documents/Master/Master/Figures/}}

% ---------------------------------------------------------------------------- %
% ---------------------------------------------------------------------------- %

\renewcommand{\comment}[1]{\ignorespaces}
%\addbibresource{/Users/morten/Documents/Master/Master/ref.bib}
\addbibresource{ref.bib}


\author{Morten Ledum}
\title{Something something computational quantum chemistry something something neural networks}
\date{December 2017}


\begin{document}
\newcommand{\mainfile}{}
\maketitle

\begin{abstract}
I havent done much worthwhile but hopefully I can bullshit my way into not failing.
\end{abstract}


%\begin{dedication}
%  To someone
%  \\\vspace{12pt}
%  This is a dedication to my cat.
%\end{dedication}

\begin{acknowledgements}
  I would like to thank my supervisors (in no particular order) Anders Malthe-Sørensen, Morten Hjorth-Jensen, and Anders Hafreager for their support during the years as a master student here at the physics department. Along with the rest of the community here at the computational physics group, you have created a wonderful environment for learning that I will miss. To Anders Hafreager, thank you for teaching me how to write code, and for helping me not fail QFT. I truly appreaciate endless enthusiasm and the willingness to go out of your way to help seemingly regardless of the nature of the problem. 

  To Morten Hjorth-Jensen, thank you for taking me under your wing here at the computational physics group. It has been a pleasure to be your student and eventual TA. I am proud to have taken some small part in teaching the introductory computational physics course with you these past few years. 

  I want to thank my tireless \emph{room mate} here in office V306, Håkon Kristiansen. It feels like we have been sitting here for decades, and I very much appreciate the company. I promise to, at some point, stop asking you difficult questions you cant answer. Maybe some day the dream will come true and someone will pay us to just sit around and \emph{figure out} whatever is interesting us that day.

  I also wish to thank Jason, the interestingly unlikely tank-driving Icelandic camel for all the stimulating distractions throughout the time working on this thesis.

  Lastly, to Vilde: Thank you for the endless encouragement (and all the sushi!). Thank you for putting up with the largely absent and massively stressed out version of myself for long months.
  \begin{flushright}
  Morten Ledum \\ Oslo, December 2017
  \end{flushright}

\end{acknowledgements}

\tableofcontents

\chapter{Introduction}
This is an intro to stuff.

\section{Molecular dynamics}
Molecular dynamics is fast but inaccurate.

\section{Quantum dynamics}
Direct quantum dynamics by solving the time dependent Schrödinger equation is accurate but unfeasably slow.

\section{Neural networks in molecular dynamics}
Neural networks can parameterize the data from millions of electronic structure calculations and in theory provide accuracy close to the direct quantum mechanical calculations at a fraction of the calculational cost.

\section{Goals}
The goals of the thesis are as follows:
\begin{itemize}
  \item World peace.
  \item Ending human suffering.
\end{itemize}

\section{Structure of the thesis}
This thesis is split into four parts. The first part, \emph{foundational theory} presents an overview of classical mechanics (briefly) and quantum mechanics as relevant for molecular dynamics and electronic structure calculations. In addition, the basics of machine learning and artifical neural networks are presented. Part one lays the groundwork and establishes most of the notation used later in part two: \emph{Advanced theory}. Here, the different approximative schemes for solving the quantum equations of motion are presented. Alongside them is the theory of molecular dynamics simulations and advanced topics in neural networks.

The penultimate part contains information w.r.t.\ the concrete implementation of frameworks described in part two. Key parts of each code base are outlined in detail. Also contained in part three are validation tests of each implementation. Lastly, results for the full work-flow utilizing all the deveolped programs are presented.

The thesis ends with part four, containing conclusions and prospect for future work.


\part{Foundational theory}

\subfile{Snippets/QuantumMechanics/QuantumMechanics}
\subfile{Snippets/WaveFunctions/WaveFunctions}

\part{Advanced theory}
\subfile{Snippets/Hartree-Fock/Hartree-Fock}
\subfile{Snippets/DFT/DFT}
\subfile{Snippets/VMC/VMC}

\part{Implementation and results}
\subfile{Snippets/VMCImplementation/VMCImplementation}
\subfile{Snippets/Hartree-FockValidation/Hartree-FockValidation}
\subfile{Snippets/VMCValidation/VMCValidation}

\part{Conclusions and future work}

\printbibliography
%\printbibliography[heading=bibintoc]
\end{document}