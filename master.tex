\documentclass[twoside,english]{uiofysmaster}
% STUFF IS LOCATED IN /Users/morten/.texmf/tex/latex/uiofysmaster
%
% Check 
% 
% https://tex.stackexchange.com/questions/96976/install-custom-cls-using-tex-live-in-local-directory
%
% for how to fix placement.

%Included packages ----------------------------------------------------------%
\usepackage[utf8]{inputenc}                  % utf-8 encoding, æ, ø , å, etc.
\usepackage{subfiles}
%\usepackage{a4wide}                          % Adjust margins to better fit A4 format.
\usepackage{array}                           % Matrices.
\usepackage{amsmath}                         % Math symbols, and enhanced matrices.o
\usepackage{amsfonts}                        % Math fonts.
\usepackage{amssymb}                         % Additional symbols.
%\usepackage{wasysym}                        % More additional symbols.
\usepackage{mathrsfs}                        % Most additional symbols.
%\usepackage[pdftex]{graphicx}                % Improved inclusion of .pdf-graphics files.
\usepackage[outercaption,wide]{sidecap}      % Floats with captions to the right/left.
\usepackage{cancel}                          % Visualize cancellations in equations.
\usepackage{enumerate}                       % Change counters (arabic, roman, etc.).
\usepackage{units}                           % Adds better looking fractions (nicefrac).
\usepackage{floatrow}                        % Multi-figure floats.
\usepackage{subfig}                          % Multi-figure floats.
\usepackage[margin=1cm]{caption}             % Adds functionality to captions.
\usepackage{bm}                              % Bolded text in math mode.
\usepackage{combinedgraphics}                % Figures; let latex handle the text itself.
\usepackage[framemethod=default]{mdframed}   % Make boxes.
\usepackage{listings}                        % For including source code.
\usepackage{soul}                            % Make vertical bars through text.
\usepackage{nicefrac}                        % Nice fractions with \nicefrac.
\usepackage{mathtools}                       % Underbrackets, overbrackets.
\usepackage{wasysym}                         % \smiley{}-s!
\usepackage{multicol}                        % Multiple text columns.
\usepackage{capt-of}                         % Caption things which are not floats.
%\usepackage[url=false]{biblatex}            % Citations (made easy).
\usepackage{dsfont}
\usepackage{booktabs}                        % Tables
\usepackage{tabularx}
\usepackage{array}
\usepackage{multirow}% http://ctan.org/pkg/multirow
\usepackage{hhline}% http://ctan.org/pkg/hhline
\usepackage{siunitx}
\usepackage[version=4]{mhchem}
\usepackage{relsize}  % Resize parts of equations
\usepackage[backend=biber,sorting=none]{biblatex}
\usepackage[version=4]{mhchem}
\usepackage{ifxetex,ifluatex}
\usepackage{etoolbox}
\usepackage{tikz}
\usepackage{framed}
\usepackage{relsize}
\usepackage[titletoc]{appendix}
\usetikzlibrary{matrix}

\captionsetup{labelfont=bf}

% Quotes --------------------------------------------------------------------- %
% conditional for xetex or luatex
\newif\ifxetexorluatex
\ifxetex
  \xetexorluatextrue
\else
  \ifluatex
    \xetexorluatextrue
  \else
    \xetexorluatexfalse
  \fi
\fi
%
\ifxetexorluatex%
  \usepackage{fontspec}
  \usepackage{libertine} % or use \setmainfont to choose any font on your system
  \newfontfamily\quotefont[Ligatures=TeX]{Linux Libertine O} % selects Libertine as the quote font
\else
  \usepackage[utf8]{inputenc}
  \usepackage[T1]{fontenc}
  \usepackage{libertine} % or any other font package
  \newcommand*\quotefont{\fontfamily{LinuxLibertineT-LF}} % selects Libertine as the quote font
\fi

\newcommand*\quotesize{60} % if quote size changes, need a way to make shifts relative
% Make commands for the quotes
\newcommand*{\openquote}
   {\tikz[remember picture,overlay,xshift=-4ex,yshift=-2.5ex]
   \node (OQ) {\quotefont\fontsize{\quotesize}{\quotesize}\selectfont``};\kern0pt}

\newcommand*{\closequote}[1]
  {\tikz[remember picture,overlay,xshift=4ex,yshift={#1}]
   \node (CQ) {\quotefont\fontsize{\quotesize}{\quotesize}\selectfont''};}

% select a colour for the shading
\colorlet{shadecolor}{listingsbackgroundcolor}

\newcommand*\shadedauthorformat{\emph} % define format for the author argument

% Now a command to allow left, right and centre alignment of the author
\newcommand*\authoralign[1]{%
  \if#1l
    \def\authorfill{}\def\quotefill{\hfill}
  \else
    \if#1r
      \def\authorfill{\hfill}\def\quotefill{}
    \else
      \if#1c
        \gdef\authorfill{\hfill}\def\quotefill{\hfill}
      \else\typeout{Invalid option}
      \fi
    \fi
  \fi}
% wrap everything in its own environment which takes one argument (author) and one optional argument
% specifying the alignment [l, r or c]
%
\newenvironment{shadequote}[2][l]%
{\authoralign{#1}
\ifblank{#2}
   {\def\shadequoteauthor{}\def\yshift{-2ex}\def\quotefill{\hfill}}
   {\def\shadequoteauthor{\par\authorfill\shadedauthorformat{#2}}\def\yshift{2ex}}
\begin{snugshade}\begin{quote}\openquote}
{\shadequoteauthor\quotefill\closequote{\yshift}\end{quote}\end{snugshade}}


% Differentials -------------------------------------------------------------- %
\newcommand{\dt}{\,\mathrm{d}t}
\newcommand{\dx}{\,\mathrm{d}x}
\newcommand{\dr}{\,\mathrm{d}r}

% Derivatives ---------------------------------------------------------------- %
\newcommand{\der} [2]{\frac{\mathrm{d} #1}{\mathrm{d} #2}}   % Derivative.
\newcommand{\pder}[2]{\frac{\partial   #1}{\partial   #2}}   % Partial derivative.

% Matrices ------------------------------------------------------------------- %
\newcommand{\mat} [2]{\begin{matrix}[#1]  #2 \end{matrix}}   % Nothing enclosing it.
\newcommand{\pmat}[2]{\begin{pmatrix}[#1] #2 \end{pmatrix}}  % Enclosing parentheses.
\newcommand{\bmat}[2]{\begin{bmatrix}[#1] #2 \end{bmatrix}}  % Enclosing square brackets.
\newcommand{\vmat}[2]{\begin{vmatrix}[#1] #2 \end{vmatrix}}  % Enclosing vertical bars.
\newcommand{\Vmat}[2]{\begin{Vmatrix}[#1] #2 \end{Vmatrix}}  % Enclosing double bars.

% Number sets ---------------------------------------------------------------- %
\newcommand{\R}{\mathbb{R}}
\newcommand{\Q}{\mathbb{Q}}
\newcommand{\N}{\mathbb{N}}
\newcommand{\Z}{\mathbb{Z}}
\newcommand{\C}{\mathbb{C}}

% Manually set alignment of rows / columns in matrices (mat, pmat, etc.) ----- %
\makeatletter
\renewcommand*\env@matrix[1][*\c@MaxMatrixCols c]{%
  \hskip -\arraycolsep
  \let\@ifnextchar\new@ifnextchar
  \array{#1}}
\makeatother

% References ----------------------------------------------------------------- %
\newcommand{\Fig}[1]{Fig.\ \ref{fig:#1}}
\newcommand{\fig}[1]{Fig.\ \ref{fig:#1}}
\newcommand{\eq} [1]{Eq.\ (\ref{eq:#1})}
\newcommand{\Eq} [1]{Eq.\ (\ref{eq:#1})}
\newcommand{\tab}[1]{Table \ref{tab:#1}}
\newcommand{\Tab}[1]{Table \ref{tab:#1}}

% Paragraph formatting ------------------------------------------------------- %
\setlength{\parindent}{5.5mm}
\setlength{\parskip}  {0mm}

% Code listings font --------------------------------------------------------- %
\lstset{basicstyle=\scriptsize}
\usepackage{ifxetex}
\ifxetex
  \usepackage{fontspec}
  \newfontfamily\listingsfontfamily[Scale=0.85]{Droid Sans Mono}
  \renewcommand{\listingsfont}{\listingsfontfamily}
\fi

\newcommand{\inlinecc}[1]{\lstinline[language={[std]c++}]{#1}}


% Convenient shorthand notation ---------------------------------------------- %
\newcommand{\nn}{\nonumber}
\newcommand{\e}[1]{\cdot10^{#1}}
\renewcommand{\i}{\hat{\imath}}
\renewcommand{\j}{\hat{\jmath}}
\renewcommand{\k}{\hat{k}}
\usepackage{ifthen}
\newcommand*\CC{\texttt{C}\kern-0.2ex\raisebox{0.2ex}{\scalebox{0.9}{+\kern-0.2ex+}}}

% Caption position of tables at the top -------------------------------------- %
\floatsetup[table]{capposition=top}

% Black frame with white background ------------------------------------------ %
\newmdenv[linecolor=black,backgroundcolor=white]{exframe}
\newmdenv[linecolor=white,backgroundcolor=shadecolor]{shadeframe}

% Including vector drawings from inkscape ------------------------------------ %
\newenvironment{combFig}[5]{
  \begin{figure}[#1] 
    \centering 
    \includecombinedgraphics[vecscale=#2, keepaspectratio]{#3} 
    \caption{#4 \label{#5}}
  \end{figure}
  }

  {
}

% Including pdf graphics ----------------------------------------------------- %
\newenvironment{pdfFig}[5]{
  \begin{figure}[#1] 
    \centering 
    \includegraphics[width= #2]{#3} 
    \caption{#4 \label{#5}}
  \end{figure}
  }

  {
}


% Set bibliography file and path for images.
\graphicspath{{./images/}}
\newcommand{\includepdfgraphics}[2]{\includecombinedgraphics[#1]{./images/#2}}


\graphicspath{{/Users/morten/Documents/Master/Master/Figures/}}

% ---------------------------------------------------------------------------- %
% ---------------------------------------------------------------------------- %

\renewcommand{\comment}[1]{\ignorespaces}
%\addbibresource{/Users/morten/Documents/Master/Master/ref.bib}
\addbibresource{ref.bib}


\author{Morten Ledum}
\title{Something something computational quantum chemistry something something neural networks}
\date{December 2017}


\begin{document}
\newcommand{\mainfile}{}
\maketitle

\begin{abstract}
I havent done much worthwhile but hopefully I can bullshit my way into not failing.
\end{abstract}


%\begin{dedication}
%  To someone
%  \\\vspace{12pt}
%  This is a dedication to my cat.
%\end{dedication}

\begin{acknowledgements}
  I would like to thank my supervisors (in no particular order) Anders Malthe-Sørensen, Morten Hjorth-Jensen, and Anders Hafreager for their support during the years as a master student here at the physics department. Along with the rest of the community at the computational physics group, you have created a wonderful environment for learning that I will sorely miss. To Morten: thank you for taking me under your wing here at the computational physics group. It has been a pleasure to be your student and later your TA. I am proud to have taken some small part in teaching the introductory computational physics course with you these past several years. 

  To my tireless \emph{room-mate} here in office V306, Håkon Kristiansen. It feels like we have been sitting here for decades, and I very much appreciate the company. I promise to stop asking you difficult questions you cant answer (at some point). Maybe one day the dream will come true and someone will pay us to just sit around and \emph{figure out} whatever is interesting to us that day.

  I want to acknowledge the two people who have probably influenced me as a student the most. Anders Hafreager and Håvard Tveit Ihle, you both are an inspiration to me, showing what you can achieve through diligent hard work. Anders, thank you for teaching me how to write (less awful) code, and for helping me not fail QFT. I truly appreaciate the endless enthusiasm and the willingness to go out of your way to help regardless of the nature of the problem. Håvard, I am looking forward to floundering my way through your cosmology lectures this spring!

  I also wish to thank the unlikely Johan, the (true[ly]) interesting Icelandic bat-camel for all the stimulating distractions throughout the time working on this thesis.

  Lastly, to Vilde: Thank you for the endless encouragement (and all the sushi!). Thank you for putting up with the largely absent and massively stressed out version of myself for long months.
  \begin{flushright}
  Morten Ledum \\ Oslo, December 2017
  \end{flushright}

\end{acknowledgements}

\tableofcontents

\chapter{Introduction}
After being deveolped and pioneered in conjunction with nuclear physics during World War II, computer modelling and numerical experiments have become ubiquitous in the natural sciences. The list of areas in which simulations have been used to produce significant new results encompass now pretty much all of them [citation needed]. Today computer simulations play as natural a part of the hard sciences as laboratory experiments and theory. In most cases all three are used in order to glean new scientific insight. ((examples here?))

%In particular, the advent of computers simulations have had a profound impact on researchers ability to 
The advent of computer simulations during the last several decades have in particular made it possible to study moderately sized quantum mechanical systems from first principles. As our ability to solve\textemdash in closed form\textemdash the governing equations of quantum mechanics (QM) vanishes extremely quickly as the number of constituent particles exceed just a few, numerics are used to augment the proverbial \emph{pen and paper}. It is striking that the underlying theory for all of chemistry and most of physics have been known for almost a century but the problem preventing us from essentially \emph{solving} chemistry is almost purely computational: the equations resulting from the exact application of this theoretical framework are way too difficult to solve.

Any approximative scheme which aims to solve the many-body Schrödinger equation from scratch subject to some (more or less) well-defined simplifications is called an \emph{ab initio} method. Working from first principles the aim of such algorithms extract information from a theoretical QM system in a reasonable amount of time. In order to accomplish this, a number of complicating intricacies need to be disregarded. The magnitude of the simplifications\textemdash essentially the number and importance of complicating factors dropped\textemdash determine both the efficacy and the efficiency of the method: More simplifications made allow solutions to be found for larger systems (albeit less precise solutions), whereas extremely precise solutions can be found for small systems if very few simplifications are employed.

Despite tremendous increases in available numerical computational power in the latter half of the previous-, and the early parts of the current century, any such approximate scheme used is still heavily limited w.r.t.\ the system size. In practice, most methods are limited to systems of containing on the order of between $10^2$ (for high-precision methods such as CI, CC, DMC, etc.) and $10^5$ electrons (for faster HF and DFT methods) \cite{hu,bowler,vandevondele}. Extracting information from larger systems neccessitate the use of classical or semi-classical algorithms, such as molecular dynamics (MD). Using MD, the time evolution of up to around $10^7$ particles can feasibly be simulated over the order of nano seconds \cite{Zhao2013,Reddy}. Corresponding large-scale cosmological $N$-body simulations have been run for as many as $10^{11}$ particles \cite{angulo,kim}. As macroscopic objects contain on the order of $10^{23}$ atoms, we are ostensibly a long way away from even classically modelling the constituent particles directly. In this region, continuum models are used.


\section{Quantum and classical dynamics}
As previously noted, solving the Schrödinger equation (SE) exactly 
by hand is impossible in the overwhelming majority of interesting cases. However, methods which can get close to the exact solution exists. Full Configuration Interaction (FCI) or direct diagonalization of the Hamiltonian is exact in the limit of an infinite orbital basis set but suffers from an exponential complexity scaling (in system \emph{and} basis size) \cite{helgaker}\comment{p525}. The related Configuration Interaction (CI) and Coupled Cluster (CC) approaches both truncate the FCI expansion of Slater determinants, thus gaining speed but loosing some accuracy \cite{kvaal,shavitt}\comment{ch6.3 ,ch 9}. Diffusion Monte Carlo (DMC) techniques can in principle provide the exact solution to the SE by imaginary-time evolution of an initial wave function guess \cite{hjorthjensen,hammond}\comment{p537,p87}. In practice, DMC methods are highly dependent on this ansatz and thus require as input the results of less accurate method but faster methods. One example may be the Variational Monte Carlo (VMC) method: conceptually simpler and faster than DMC, but not as accurate \cite{hammond,conroy,anderson}.

The Hartree-Fock (HF) framework\textemdash which provides an efficient but not enormously accurate result\textemdash has seen extensive use since its inception in 1930 \cite{hartree,fock,szabo}. However, by far the most popular approximation is Density Functional Theory (DFT), developed by W. Kohn and L. J. Sham in 1965 \cite{kohnsham,martin}. Between 1980 and 2010, DFT was the most active field in physics with eight out of the top ten most cited papers being on the subject \cite{dftperspective}.


"This is where multiscale modeling comes in. By considering simultaneously models at different scales, we hope to arrive at an approach that shares the efficiency of the macro- scopic models as well as the accuracy of the microscopic models." Weinan: Prinicples of Multiscale modelling side viii

\section{Neural networks in molecular dynamics}
Neural networks can parameterize the data from millions of electronic structure calculations and in theory provide accuracy close to the direct quantum mechanical calculations at a fraction of the calculational cost.

\section{Goals}
The goals of the thesis are as follows:


\section{Structure of the thesis}
This thesis is split into four parts. The first part, \emph{foundational theory} presents an overview of classical mechanics (very briefly) and quantum mechanics as relevant for molecular dynamics and electronic structure calculations. In addition, the basics of machine learning and artifical neural networks are presented. Part one lays the groundwork and establishes most of the notation used later in part two: \emph{Advanced theory}. Here, the different approximative schemes for solving the quantum equations of motion are presented. Alongside them is the theory of molecular dynamics simulations and advanced topics in neural networks.

The penultimate part contains information w.r.t.\ the concrete implementation of frameworks described in part two. Key parts of each code base are outlined in detail. Also contained in part three are validation tests of each implementation. Lastly, results for the full work-flow utilizing all the deveolped programs are presented.

The thesis ends with part four, containing conclusions and prospect for future work.


\part{Foundational theory}

\subfile{Snippets/QuantumMechanics/QuantumMechanics}
\subfile{Snippets/WaveFunctions/WaveFunctions}

\part{Advanced theory}
\subfile{Snippets/Hartree-Fock/Hartree-Fock}
\subfile{Snippets/DFT/DFT}
\subfile{Snippets/VMC/VMC}

\part{Implementation and results}
\subfile{Snippets/HFImplementation/HFImplementation}
\subfile{Snippets/VMCImplementation/VMCImplementation}
\subfile{Snippets/Hartree-FockValidation/Hartree-FockValidation}
\subfile{Snippets/VMCValidation/VMCValidation}

\part{Conclusions and future work}


\begin{appendices}
\appendixpage
\noappendicestocpagenum
\addappheadtotoc
\subfile{Snippets/Appendix/Appendix}
\end{appendices}


\printbibliography
%\printbibliography[heading=bibintoc]
\end{document}